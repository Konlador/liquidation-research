%%%%%
%%%%%  Naudokite LUALATEX, ne LATEX.
%%%%%
%%%%
\documentclass[]{VUMIFTemplateClass}

\usepackage{indentfirst}
\usepackage{amsmath, amsthm, amssymb, amsfonts}
\usepackage{mathtools}
\usepackage{physics}
\usepackage{graphicx}
\usepackage{verbatim}
\usepackage[hidelinks]{hyperref}
% \usepackage{color,algorithm,algorithmic}
\usepackage[nottoc]{tocbibind}
\usepackage{tocloft}
\usepackage{makecell}
\usepackage{color}
\usepackage{titlesec}
\usepackage{algorithmicx}
\usepackage{algpseudocode}
\usepackage{bm}
\usepackage{caption}
\usepackage{float}
\usepackage{listings}
\usepackage{subfig}
\usepackage{wrapfig}
\usepackage{enumitem}

\newcommand{\sectionbreak}{\clearpage}

\makeatletter
\renewcommand{\fnum@algorithm}{\thealgorithm}
\makeatother
\renewcommand\thealgorithm{\arabic{algorithm} algoritmas}

\usepackage{biblatex}
\bibliography{bibliografija}
%% norint pakeisti bibliografijos šaltinių numeravimą (skaitiniu arba raidiniu), pakeitimus atlikti VUMIFTemplateClass.cls 150 eilutėje

% Author's MACROS
\newcommand{\EE}{\mathbb{E}\,} % Mean
\newcommand{\ee}{{\mathrm e}}  % nice exponent
\newcommand{\RR}{\mathbb{R}}


\studijuprograma{Programų sistemų}
\darbotipas{Bakalauro baigiamasis darbas}
\darbopavadinimas{Likvidavimo algoritmo tobulinimas perviršinio užstato skolinimo protokoluose}
\darbopavadinimasantras{Improving liquidation algorithms in over-collateralized lending protocols}
\autorius{Vismantas Stonkus}
\vadovas{prof. dr. Remigijus Paulavičius}
\recenzentas{lekt. Karolis Uosis}

\begin{document}
\selectlanguage{lithuanian}

\singlespacing
\begin{titlepage}
\vskip 20pt
\begin{center}
\includegraphics[scale=0.55]{images/MIF.png}
\end{center}

\makeatletter

\vskip 20pt
\centerline{\bf \large \textbf{VILNIAUS UNIVERSITETAS}}
\vskip 10pt
\centerline{\large \textbf{MATEMATIKOS IR INFORMATIKOS FAKULTETAS}}
\vskip 10pt
\centerline{\large \textbf{\MakeUppercase{\@studijuprograma \space studijų programa}}}

\vskip 80pt
\centerline{\Large \@darbotipas}
\vskip 20pt
\vskip 80pt

\centering{
    \begin{tabular}{rcp{.7\textwidth}}
        {\Large Praktiką atliko} & {\Large :} & {\Large \@autorius}\\[10pt]
        {\Large Universiteto praktikos vadovas} & {\Large :} & {\Large \@vadovas}\\[10pt]
        {\Large Praktikos institucija} & {\Large :} & {\Large \@institucija}\\[10pt]
        {\Large Organizacijos praktikos vadovas} & {\Large :} & {\Large \@institucijosVadovas}\\[10pt]
        {} & {} & {\Large Technologijų vadovas (CTO)}\\[10pt]
        {\Large Organizacijos praktikos vadovo įvertinimas} & {\Large :} & {\Large \@institucijosIvertinimas}\\[10pt]
    \end{tabular}}

\vskip 110pt

\centerline{\large \textbf{Vilnius}}
\centerline{\large \textbf{\the\year{}}}

\makeatother

\newpage
\end{titlepage}
%\newgeometry{top=2cm,bottom=2cm,right=2cm,left=3cm}
\setcounter{page}{2}


\tableofcontents

\onehalfspacing

%%Santrauka
\selectlanguage{lithuanian}
\sectionnonum{Santrauka}
Darbe nagrinėjamas perviršinio užstato skolinimo protokolų likvidavimo procesas, siekiant padidinti likvidatoriaus pelningumą. Analizei pasirinktas \textit{Venus} protokolas, veikiantis \textit{Binance Smart Chain} tinkle. Pateikiama kelių algoritminių strategijų apžvalga, leidžiančių optimizuoti paskolos grąžinimo ir užstato paėmimo pasirinkimą, siekiant padidinti likvidacijos efektyvumą. Be klasikinės „iki uždarymo ribos“ strategijos, darbe nagrinėjama „pilno išeikvojimo“ strategija, kurios tikslas – maksimaliai išnaudoti skolininko poziciją nepažeidžiant likvidacijos apribojimų. Taip pat suformuluotos keturios papildomos metodikos: „didžiausios skolos“, „pilnas išeikvojimas vienodoų valiutų“, „nuo didžiausio užstato koeficiento“ ir „nuo mažiausio užstato koeficiento“. Strategijų veiksmingumas vertinamas remiantis istoriniais duomenimis.
\textcolor{red}{Toliau tyrimo išvados.}


\noindent\textbf{Raktiniai žodžiai:} blokų graindinės, decentralizuoti finansai, likvidacija, arbitražas, maksimaliai išgaunama vertė, skolinimo protokolai.

\sectionnonum{Summary}
\textcolor{red}{Išversti}
% The study examines ways to optimize liquidation processes in over-collateralized lending protocols. The research focuses on analyzing liquidation strategies and propose algorithmic modifications to maximize the liquidator’s profit. A review of the existing methods includes a detailed analysis of the “Optimal Fixed Spread Liquidation Strategy” algorithm. A proposed modification, called "the drain" strategy, optimizes the process, enabling full utilization of the liquidation opportunity without losing its potential profit. Empirical evidence from the Venus protocol shows that most liquidations do not achieve the maximum possible profit. A comparison of three strategies (“repeat,” “up to close factor,” and “drain”) shows that "the drain" strategy generate up to 72\% more profit than historical liquidations. This demonstrates that when optimized strategies can significantly increase liquidators’ profitability. The findings of this study contribute to the ecosystem of lending protocol optimization and pave the way for further analyses related to currency pair and collateral management strategies. Future research could explore using different currencies and other liquidation opportunities to enhance the efficiency of arbitrage algorithms.\\

\textbf{Keywords:} blockchain, liquidation, arbitrage, mev (maximal extractable value).


%Sutrumpinimų skyrius
\sectionnonum{Santrumpos}
\begin{tabular}{rcp{.7\textwidth}}
    {DeFi} & {} & {decentralizuotų finansų sistema, kurioje tradicinės finansinės paslaugos (paskolos, skolinimas, keitimas, draudimas ir kt.) teikiamos naudojant išmaniuosius kontraktus blokų grandinėse}
\end{tabular}

\sectionnonum{Įvadas}
% \begin{enumerate}
%     \item apibrėžiamas tiriamasis objektas akcentuojant neapibrėžtumą, kuris bus išspręstas darbe,
%     \item aprašomas temos aktualumas,
%     \item nurodomas darbo tikslas ir uždaviniai, kuriais bus įgyvendinamas tikslas.
%     \item aptariamos teorinės darbo prielaidos bei kokios metodikos ir kuriam tikslui naudojamos,
%     \item apibūdinami su tema susiję literatūros ar kitokie šaltiniai,
%     \item temos analizės tvarka, darbo atlikimo aplinkybės,
%     \item pateikiama žinių apie naudojamus instrumentus (programas ir kt., jei darbe yra eksperimentinė dalis). Darbo įvadas neturi būti dėstymo santrauka. 
% \end{enumerate}

% Darbo uždavinyje\\
% apibrėžiamas siekiamas rezultatas, kad būtų galimybė išmatuoti, ar tikslai ir
% uždaviniai yra išspręsti, bei kokiu lygiu (vertinant kiekybę bei kokybę). Pavyzdžiui, „Atlikti literatūros
% .... analizę“ nėra tinkamas uždavinys, nes nusako procesą, tačiau neapibrėžia jo rezultato. Tinkamos
% uždavinio formuluotės šablonai: „Išanalizuoti literatūrą … siekiant apžvelgti ir įvertinti /… metodų
% tinkamumą sprendžiamam uždaviniui/privalumus ir trūkumus sprendžiant … uždavinį/rekomenduojamas ... projektavimo gaires, šablonus ir pan.“

\subsection*{Tyrimo objektas ir aktualumas}
Šiame darbe tiriamas decentralizuotų kriptovaliutų paskolų protokolų likvidavimo mechanizmas, ypatingą dėmesį skiriant \textit{Venus} protokolui, veikiančiam \textit{Binance Smart Chain} (BSC) tinkle. Nagrinėjamas būdas, kaip galima padidinti likvidatoriaus pelningumą per likvidacijos optimizavimą. Kadangi \textit{Venus} veikimo principas yra panašus į kitų decentralizuotų skolinimo sistemų, šio darbo rezultatai turi platesnį pritaikomumą.

Šios temos aktualumą lemia augantis susidomėjimas decentralizuotomis finansų sistemomis (DeFi), kuriose veikiantys likvidavimo mechanizmai atveria naujas pelno gavimo galimybes. Viena iš pagrindinių tokio pelno formų – arbitražas, kai likvidatorius siekia pasinaudoti sistemos teikiamomis paskatomis grąžinti skolas mainais į vertingesnį užstatą. Norint užtikrinti maksimalų efektyvumą, būtina gebėti modeliuoti ir optimizuoti likvidacijų eigą. Todėl šios temos analizė aktuali tiek individualiems likvidatoriams, norintiems padidinti pelną, tiek protokolų kūrėjams, siekiantiems užtikrinti teisingą ir efektyvų sistemos veikimą.

\subsection*{Darbo tikslas ir uždaviniai}
Darbo tikslas – sukurti ir optimizuoti likvidavimo algoritmą, kuris maksimaliai padidintų likvidatoriaus pelną.

Norint įgyvendinti darbo tikslą, išsikelti šie uždaviniai:
\begin{enumerate}
    \item Apžvelgti esamus perviršinio užstato skolinimo protokolus, išanalizuoti jų veikimo principus ir palyginti juos su Venus protokolu.
    % Kadangi daugelis skolinimo protokolų kriptovaliutų ekosiste moje veikia panašiai, jei ši prielaida pasitvirtins, gauti rezultatai galės būti pritaikyti ir kituose protokoluose.
    \item Išsamiai išnagrinėti \textit{Venus} protokolo likvidavimo mechanizmą.
    \item Sukurti ir/arba modifikuoti efektyvesnį likvidavimo algoritmą. Šiuo uždaviniu pristatome likvidavimo strategijas:
    \begin{itemize}
    \item \textbf{Iki uždarymo ribos} (angl. \textit{up to close factor}) – grąžinama paskolos suma iki maksimalios leidžiamos ribos pagal protokolo uždarymo ribos parametrą, kai skolos ir užstato valiutų pora yra iš anksto fiksuota.
        \item \textbf{Pilnas išeikvojimas} (angl. \textit{drain}) – strategija, vykdanti kelias iš eilės likvidacijas, siekiant maksimaliai išeikvoti skolininko poziciją nepažeidžiant likvidacijos ribojimų. Strategija pilnai išnaudoja tik vieną konkrečią skolos ir užstato valiutų porą.
      \item \textbf{Didžiausia skola} (angl. \textit{single largest borrow}) – grąžinama ta skolos valiuta, kurios suma yra didžiausia. Užstatas pasirenkamas iš tos pačios valiutos, jei užstato kiekis yra pakankamas, kitu atveju – valiuta, kurios vertė skolininko portfelyje didžiausia.
      \item \textbf{Nuo didžiausio užstato koeficiento} (angl. \textit{from largest collateral factor}) – pirmiausia pasirenkamos užstato valiutos su didžiausiu užstato koeficientu, o paskutinė likvidacija vykdoma pagal „Didžiausios skolos“ strategiją.
      \item \textbf{Nuo mažiausio užstato koeficiento} (angl. \textit{from smallest collateral factor}) – analogiška ankstesnei strategijai, tačiau prioritetas teikiamas užstato valiutoms su mažiausiu užstato koeficientu.
      \item \textbf{Vienodos valiutos} (angl. \textit{same tokens}) – atliekamos „Pilno išeikvojimo“ strategijos likvidacijos tik toms valiutoms, kurios yra tiek užstatytos, tiek pasiskolintos. Ši strategija leidžia sumažinti valiutų keitimo riziką ir likvidumo problemas.
    \end{itemize}
    \item Palyginti sukurtas strategijas tarpusavyje bei su istorinėmis likvidacijomis.
    % \item Pasiūlyti, aprašyti ir/arba modifikuoti likvidacijos algoritmus, orientuotus į skolos ir užstato valiutų pasirinkimo strategijas.
    % \item Kiekybiškai įvertinti strategijų pelningumą remiantis istoriniais duomenimis ir kuro sąnaudomis.
    % \item Palyginti gautus rezultatus su istorinėmis likvidacijomis ir pateikti išvadas.
\end{enumerate}

\subsection*{Metodai}
Empirinei analizei naudotas \textit{Venus} protokolo kodas bei istoriniai duomenys iš BSC tinklo. Strategijų simuliacija įgyvendinta naudojant \textit{Forge} testavimo sistemą. Analizei naudoti duomenys surinkti naudojant \textit{quicknode.com} archyvinę prieigą prie BSC blokų grandinės.

Naudojamas supaprastintas pelno skaičiavimo modelis, kuris remiasi tuo metu galiojančiomis orakulo kainomis ir faktinėmis strategijos kuro sąnaudomis, tačiau nenumato privalomo pelno konvertavimo į pradinę valiutą. Tai leidžia išvengti papildomų arbitražo veiksmų analizės ir koncentruotis tik į pačios likvidacijos pelningumą.

\subsection*{Literatūra}
\textcolor{red}{
Išsiaiškinti kas tiksliau čia turi būti aprašyta.
}

Literatūros apžvalgoje remiamasi tiek technine dokumentacija, tiek moksliniais darbais apie likvidavimo mechanizmus ir decentralizuotas finansų sistemas.
% Taip pat naudojami oficialūs protokolų dokumentai (pvz., Venus ar Aave techninė dokumentacija) ir istoriniai duomenų šaltiniai, leidžiantys atkurti realias likvidacijas.

\subsection*{Temos analizės tvarka, darbo atlikimo aplinkybės}
\textcolor{red}{
???
}
% Darbo pradžioje pateikiamas arbitražo sąvokos paaiškinimas, kad būtų suprasta likvidacijos ekonominė motyvacija. Vėliau aptariamas pats likvidavimo mechanizmas, jo ypatybės DeFi kontekste. Toliau pateikiamos ir formalizuojamos kelios likvidacijos strategijos, kurios vėliau empiriškai palyginamos naudojant realių istorinių likvidacijų duomenis. Galiausiai pateikiami rezultatai ir įžvalgos apie skirtingų strategijų efektyvumą.

\section{Kas yra arbitražas}
Arbitražo sąvokos išmanymas yra svarbus šio darbo kontekste, nes būtent arbitražo galimybės motyvuoja likvidatorių veiklą decentralizuotose finansų (DeFi) sistemose. Likvidacijos metu gautą užstatą dažnai reikia konvertuoti į kitą valiutą siekiant užfiksuoti pelną, o tai yra arbitražui būdingas veiksmas. Todėl norint suprasti, kaip optimizuoti likvidacijų pelningumą, būtina suvokti arbitražo principus. Šis skyrius apibrėžia, kas yra arbitražas, ir pateikia pagrindines jo formas bei realizavimo galimybes kriptovaliutų rinkose.

Arbitražas yra finansinė strategija, grindžiama ekonominių skirtumų išnaudojimu tarp dviejų ar daugiau rinkų, siekiant gauti pelną be jokios rizikos ar su minimalia rizika. Paprastai tariant, tai yra praktika, kai investuotojai perka tam tikrus finansinius instrumentus, tokius kaip prekės ar valiutos pigesnėje rinkoje, ir tuoj pat pardavinėja juos brangesnėje rinkoje, gaudami pelną iš šio kainų skirtumo.

Vienas esminių arbitražo principų yra reliatyvumas -- investuotojai renkasi referencinę valiutą, kurios vertę jie laiko stabilia ir dirba siekdami padidinti šios valiutos kiekį. Pavyzdžiui, jei arbitražininkas laiko eurą stabilia valiuta, jis perka ir parduoda kitas valiutas arba kitokį turtą, tikėdamasis gauti kuo didesnį pelną eurais.

Paprasčiausias arbitražas susideda iš kainų skirtumų tarp dviejų rinkų. Tai pasiekiama konvertuojant stabilų turto vienetą į kitą valiutą pagal kursą $P_1$ ir vėliau konvertuojant šią valiutą atgal į pradinį stabilų turto vienetą kursu $P_2$, kur $P_1 \cdot P_2 > 1$. Šią pagrindinę idėją galima plėtoti ir modifikuoti įvairiais būdais:

\begin{itemize}
    \item įtraukti tarpinius konvertavimo žingsnius, sudarant ilgesnę operacijų grandinę, siekiant išnaudoti kainų skirtumus biržose, kuriose nėra tiesioginio konvertavimo kelio iš ar į stabiliąją valiutą;
    \item pradėti arbitražą ne vien tik su viena stabiliąja valiuta, bet su keliomis. Taip galima efektyviau išnaudoti turimą kapitalą, nes skirtingos stabilių valiutų pozicijos gali suteikti daugiau lankstumo ir galimybių reaguoti į kainų skirtumus įvairiose biržose;
    \item išskaidyti valiutos konvertavimą į kitą per kelias biržas, siekiant išvengti didelio kainų svyravimo vienoje biržoje;
    \item konvertuoti vieną turto vienetą į kelias skirtingas valiutas vienoje tranzakcijoje dėl tam tikrų platformų galimybių;
    \item pasiskolinti kapitalą iš biržos su sąlyga, kad skola bus grąžinta per arbitražo procesą, taip panaikinant reikalavimą turėti pradinį kapitalą.
\end{itemize}

Efektyviausia arbitražo ypatumus pateikti pasinaudojant grafais. Pavyzdžiui, grafe, kur mazgai reprezentuoja savininkų sąskaitas (tradicinių valiutų atveju) arba piniginės adresus (kriptovaliutų atveju), o kryptinės briaunos – valiutų ar kriptovaliutų persiuntimus tarp jų, aiškiai matoma turtų judėjimo dinamika ir galimos arbitražo pelno galimybės. 

\begin{figure}[H]
    \centering
    \includegraphics[scale=0.3]{img/arb1.png}
    \caption{Arbitražo vizualizacija: lėšų pervedimai ir galutinis pelnas – 0,11899 WBNB}
    \label{img:chess-minimax-2}
\end{figure} 

\ref{img:chess-minimax-2} pav. pateikta arbitražo vizualizacija kriptovaliutų atveju. Matome 12 pavedimų tarp skirtingų piniginių adresų (numeracija prasideda nuo 0). Šiuo atveju adresas 0x85a..aab yra pradinis kontrakto adresas, kuris buvo iškviestas tranzakcijoje. Iš viso tranzakcijoje dalyvauja 4 rinkos, kuriose galima konvertuoti valiutų poras:

\begin{itemize}
    \item 0x8eo..706 -- CUBI - WBNB
    \item 0xf55..a78 -- CUBI - USDT
    \item 0xcbc..62c -- USDT - WBNB
    \item 0xd64..73a -- USDT - WBNB
\end{itemize}

Adresas 0x4ad..00c šiuo atveju yra tik pelno piniginė, kurioje tranzakcijos pabaigoje atsiranda papildomi 0,11899 WBNB valiutos.

Kiekviename mazge taip pat pažymėtos valiutos ir jų kiekiai, nurodantys galutinį valiutos skirtumą po tranzakcijos. Pavyzdžiui, rinkoje 0x8eo..706 valiuta WBNB -> CUBI buvo konvertuota du kartus, kur iš viso į rinką įdėta 2,235993 WBNB, o išimta 13748,29 CUBI.

Tranzakciją galima suskirstyti į dvi dalis su pavedimais 0-5 ir 6-11. Pirmuosiuose penkiuose pavedimuose (0-4) atliekamas ratas: pradžioje trumpam pasiskolinama CUBI, tada ji iškeičiama į USDT, o po to -- į WBNB. Trečiuoju pavedimu gaunama daugiau valiutos, nei reikia skolai grąžinti ketvirtuoju pavedimu. Todėl perviršis lieka kaip pelnas, kuris penktuoju pavedimu persiunčiamas į pelno piniginę. Panašiai vyksta ir su pavedimais 6-11, tačiau antrojoje dalyje naudojama kita USDT-WBNB rinka, o apskritai keičiamų valiutos kiekių dydžiai yra mažesni.

\section{Likvidacijos}

Šiame skyriuje detaliai nagrinėjamas likvidavimo mechanizmas decentralizuotuose skolinimo protokoluose. Likvidacijos procesas yra esminė perviršinio užstato skolinimo platformų dalis, užtikrinanti jų stabilumą ir saugumą. Jo supratimas yra būtinas norint įvertinti, kaip automatizuoti algoritmai gali efektyviai išnaudoti šį mechanizmą pelno generavimui. Ši tema svarbi ne tik dėl praktinio aktualumo likvidatoriams, bet ir todėl, kad tai išskiria DeFi sistemų veikimą nuo tradicinių finansinių platformų. Skiriant daugiau dėmesio šiam procesui, galima tiksliau suformuluoti optimizavimo tikslus bei pagrįsti siūlomų strategijų svarbą.
% Koks yra sios analizes tikslas? Suprasti apie defi paskolu platformas ir kuo ju likvidavimo procesas issiskiria nuo tradiciniu finansu sistemu.


\subsection{Tradicinės paskolų platformos}
Paskolų platformos – tai sistemos, kuriose vartotojai gali skolintis ar skolinti pinigus kitiems vartotojams. Tradicinėse bankinėse sistemose bankas veikia kaip tarpininkas tarp skolintojo ir skolininko. Panašiai veikia ir kredito unijos, kuriose nariai vieni kitiems suteikia paskolas per bendrą fondą.

\subsubsection{Veiklos principas}
\begin{itemize}
    \item \textbf{Skolintojai}: asmenys ar įmonės, turintys perteklinių lėšų, kurias jie norėtų investuoti, kad gautų grąžą kaip palūkanas.
    \item \textbf{Skolininkai}: asmenys ar įmonės, kuriems reikia lėšų įvairiems tikslams – pradėti verslą, finansuoti projektą, pirkti prekę ir pan.
\end{itemize}

Skolintojai įneša savo lėšas į platformą, o skolininkai pateikia prašymus dėl paskolų. Platforma tada suderina šiuos du suinteresuotus šalių poreikius. Priklausomai nuo platformos ir paskolos tipo, gali būti reikalaujama, kad skolininkai užstatytų tam tikrą turto dalį kaip garantiją.

\subsubsection{Likvidacijos tradicinėse sistemose}
Tradicinėse finansų sistemose likvidacija vyksta tuomet, kai skolininkas nesugeba įvykdyti savo įsipareigojimų grąžinti paskolą. Tokiu atveju kredito įstaiga (bankas ar kredito unija) gali pradėti priverstinį skolų išieškojimo procesą, kuris dažnai apima užstato realizavimą.

Procesas dažnai prasideda nuo skolininko įspėjimo apie neįvykdytą mokėjimą. Jeigu situacija nesikeičia, kreditorius gali inicijuoti teisinį procesą: kreiptis į antstolius arba teismą dėl turto arešto ir pardavimo. Užstatas, pavyzdžiui, nekilnojamasis turtas ar transporto priemonė, įvertinamas ir parduodamas aukcione arba per kitus kanalus.

Gautos lėšos naudojamos skolos padengimui. Jeigu užstato realizavimo suma viršija skolos vertę, likutis grąžinamas skolininkui. Priešingu atveju, jei skolos likutis išlieka, skolininkas lieka skolingas likusią sumą. Procesas yra ilgesnis nei automatizuotose sistemose, tačiau suteikia galimybę ginčams spręsti ir apsaugoti abi puses teisinėmis priemonėmis.

\subsection{Kriptovaliutų paskolų platformos}
Šiuolaikinėse paskolų platformose populiarėja kriptovaliutų paskolų modelis, kuriame operacijos atliekamos naudojant kriptovaliutas arba kitą kriptoturtą kaip užstatą. Skolinimo ir grąžinimo procesai šiose platformose vykdomi naudojant išmaniuosius kontraktus (angl. \textit{smart contracts}). Tokios platformos priklauso decentralizuotų finansų (angl. \textit{Decentralized Finance}, \textit{DeFi}) sistemai – tai finansinių paslaugų ekosistema, veikianti be tarpininkų, pagrįsta blokų grandinės technologija ir automatizuota išmaniųjų kontraktų pagalba.

Viena iš svarbiausių šių platformų savybių yra tai, kad skolininkai privalo pateikti užstatą, kuris viršija paimtos paskolos vertę. Kriptovaliutų pasaulyje tokia perviršinio užstato sistema yra būtina, nes leidimas skolintis daugiau nei pateikto užstato vertė atvertų kelią reikšmingam išnaudojimui. Jei būtų galima pateikti mažai užstato ir gauti didelę paskolą, tai sukeltų didelę riziką platformai, nes skolininkai galėtų nesąžiningai išnaudoti šią galimybę, o skolintojai patirtų didžiulius nuostolius. Todėl perviršinio užstato reikalavimas yra esminis saugumo mechanizmas, užtikrinantis stabilumą ir patikimumą šiose finansinėse sistemose.\cite{whatisdefiliquidation}

Dar viena kriptovaliutų paskolų platformų ypatybė -- tai valiutų vertės nustatymo procesas, kuris yra atliekamas naudojant orakulus. Orakulai yra išoriniai informacijos tiekėjai, kurie pateikia duomenų srautą naudojantis išmaniaisiais kontraktais. Šie kontraktai yra periodiškai atnaujinami per privilegijuotas \textit{blockchain} tranzakcijas. Šios tranzakcijos nustato tam tikros valiutos vertę, užtikrindamos sklandų ir patikimą platformos veikimą.

\subsubsection{Kodėl žmonės skolinasi per kriptovaliutų paskolų platformas?}

\begin{enumerate}
    \item \textbf{Kapitalo prieinamumas neparduodant turto}: skolinantis per kriptovaliutų platformas, galima gauti lėšų neparduodant turimų kriptovaliutų, taip išvengiant galimų mokesčių ir toliau dalyvaujant rinkos augime \cite{kriptovaliutosio}.
    \item \textbf{Greitas ir paprastas procesas}: kriptovaliutų paskolos dažnai suteikiamos greičiau ir su mažiau biurokratijos nei tradicinės bankų paskolos, nes nereikalaujama kredito istorijos patikrinimo \cite{targettrend}.
    \item \textbf{Trumpasis pardavimas}: dar angliškai vadinamas \textit{shorting}. Kriptovaliutų skolinimo platformos leidžia naudotojams skolintis kriptovaliutas, kurias jie vėliau parduoda dabartine rinkos kaina tikėdamiesi kainų kritimo ateityje. Kainai nukritus yra nuperkamas tas pats kiekis valiutos, kiek buvo pasiskolinta, ir iš karto gražinama skola. Šis metodas leidžia investuotojams uždirbti iš rinkos kainų svyravimų net ir krintant kainoms. Pavyzdžiui, investuotojas gali užstatyti savo ETH, pasiskolinti BTC, parduoti BTC už \$30000, tariant, kad einamuoju metu yra tokia kaina, ir vėliau nusipirkti už \$25000, tariant, kad po kažkiek laiko kaina nukrito, grąžindamas skolą ir pasilikdamas \$5000 pelno.
    \cite{shortinimas}.
\end{enumerate}

\subsubsection{DeFi likvidacijų palyginimas su tradicinę finansų sistemą}
DeFi likvidacijos procesas skiriasi nuo tradicinio likvidavimo keliais esminiais aspektais, kurie yra svarbūs norint suprasti, kodėl šiose sistemose leidžiama įtraukti daugiau dalyvių – trečiąsias šalis, atliekančias likvidaciją\cite{whatisdefiliquidation}:

\begin{itemize}
    \item \textbf{Automatizavimas:} DeFi likvidacijos paprastai yra automatizuotos naudojant išmaniuosius kontraktus, todėl nereikia žmogaus įsikišimo. Tradicinėse finansų sistemose maržos reikalavimas (angl. \textit{margin call}) dažnai reikalauja skolininko ar tarpininko rankinio veiksmo.
    
    \item \textbf{Skaidrumas:} DeFi likvidacija yra visiškai skaidri – visos tranzakcijos vyksta viešoje blokų grandinėje. Tradicinėse finansų sistemose likvidacijos detalės gali būti sunkiau prieinamos ar nematomos plačiajai visuomenei.
    
    \item \textbf{Decentralizacija:} DeFi veikia be tarpininkų – kriptovaliutų likvidacijos vykdomos tiesiogiai per kodo logiką. Tradicinėse finansų sistemose likvidacijas valdo brokeriai arba finansinės institucijos.
\end{itemize}

Kai skolininko užstato vertė krinta žemiau likvidavimo ribos (angl. \textit{liquidation threshold}), pagal protokolą bet kam leidžiama atlikti likvidaciją. Per likvidacijos procesą užstatas (angl. \textit{collateral}) yra parduodamas, dažnai su tam tikra nuolaida, mainais į valiutą, kurią skolininkas pasiskolino. \cite{venusliquidations}

\subsubsection{Protokolai}

Yra daugybė paskolų protokolų, veikiančių įvairiuose \textit{blockchain} tinkluose, kurie suteikia vartotojams galimybę skolintis ir skolinti decentralizuotai. \ref{tab:sample_table} lentelėje pateikiami keli populiariausi šių protokolų pavyzdžiai. 

\begin{itemize}
  \item \textbf{Blokų grandinės} – išvardijami \textit{blockchain} tinklai, kuriuose veikia atitinkamas protokolas. Paskolų mechanizmas tarp tinklų gali skirtis dėl tinklo subtilybių, tačiau pagrindiniai procesai išlieka tokie pat tam pačiam protokolui.
  \item \textbf{Bendra užrakinta vertė (Total Value Locked)} – atspindi platformos bendrą turto vertę įdėtą naudotojų. Tai svarbus rodiklis, leidžiantis įvertinti protokolo populiarumą ir pasitikėjimą. Paskolų platformų kontekste, ši vertė nurodo bendrą užstato sumą, iš kurios kaupiamos palūkanos už paskolintą turtą.
  \end{itemize}
  
Iš lentelės matome, kad Aave yra vienas iš populiariausių ir labiausiai paplitusių paskolų protokolų, veikiantis daugybėje \textit{blockchain} tinklų ir turintis didžiausią bendrą užrakintą vertę – \$22,112 milijardų. Šis protokolas išsiskiria savo plačia integracija ir stipriu vartotojų pasitikėjimu.


\begin{table}[H]
  \centering
  \caption{Paskolų protokolų palyginimas \cite{LikvidacijuProtokolai}}
  \begin{tabular}{|l|p{7cm}|p{5cm}|}
  \hline
  \textbf{Pavadinimas}       & \textbf{Blokų grandinės}                                                                 & \makecell{\textbf{Bendra užrakinta vertė} \\ \textbf{(Total Value Locked)}} \\ \hline
  Aave                      & Ethereum, Arbitrum, Avalanche, Polygon, Base, Optimism, Scroll, BSC, Gnosis, Metis, ZKsync Era, Fantom, Harmony & \$22,112B \\ \hline
  JustLend                  & Tron                                                                                  & \$7,282B  \\ \hline
  Spark                     & Ethereum, Gnosis                                                                      & \$5,391B  \\ \hline
  Compound Finance          & Ethereum, Arbitrum, Base, Optimism, Polygon, Scroll                                   & \$3,051B  \\ \hline
  Morpho                    & Ethereum, Base                                                                        & \$2,930B  \\ \hline
  Kamino Lend               & Solana                                                                                & \$2,126B  \\ \hline
  Venus                     & BSC, ZKsync Era, Arbitrum, opBNB, Ethereum                                            & \$1,938B  \\ \hline
  Avalon Labs               & CORE, Bitlayer, Taiko, BSC, Merlin, BOB, Mode, BSquared, ZetaChain, Kaia, Arbitrum, Ethereum, Base, IoTeX, Scroll, Mantle, Zircuity & \$1,134B  \\ \hline
  Fluid Lending             & Ethereum, Arbitrum, Base                                                              & \$622,38M \\ \hline
  Benqi Lending             & Avalanche                                                                             & \$511,67M \\ \hline
  NAVI Lending              & Sui                                                                                   & \$476,31M \\ \hline
  Suilend                   & Sui                                                                                   & \$468,55M \\ \hline
  Marginfi Lending          & Solana                                                                                & \$453,06M \\ \hline
  \end{tabular}
  \label{tab:sample_table}
\end{table}

\subsubsection{Venus platforma}

% Šiame darbe toliau yra analizuojamas \textit{Venus} protokolas ir jo istorinės likvidacijos. Nors ši platforma nėra didžiausia, jos likvidavimo mechanizmai yra panašūs į daugelio kitų kriptovaliutų paskolų platformų, todėl darbo rezultatai gali būti laikomi reprezentatyviais platesnei šios srities ekosistemai. \textit{Venus} protokolas pasirinktas dėl autoriaus asmeninio susipažinimo su jo veikimu ir ilgametės patirties pagrindiniame jo \textit{blockchain} tinkle – \textit{Binance Smart Chain (BSC)}.

Šiame darbe toliau yra analizuojamas \textit{Venus} protokolas ir jo istorinės likvidacijos. Nors ši platforma nėra didžiausia, jos likvidavimo mechanizmai yra panašūs į daugelio kitų kriptovaliutų paskolų platformų. Tai bus detaliau parodyta skyriuje \ref{sec:venus_mechanizmas}

\textit{Venus} protokolas pasirinktas dėl autoriaus asmeninio susipažinimo su jo veikimu ir ilgametės patirties pagrindiniame jo \textit{blockchain} tinkle – \textit{Binance Smart Chain (BSC)}.

\subsubsection{Arbitražo potencialas}
Iš esmės, likvidaciją galima laikyti tam tikra rinka, suteikiančia galimybę konvertuoti vieną valiutą į kitą. Likvidacija paprastai yra pelninga likviduotojams, nes pasiskolinta valiuta yra iškeičiama į užstatą, kurio vertė dažnai viršija grąžintos paskolos sumą – taip likvidatoriams sukuriama finansinė paskata veikti. Ši procedūra yra nepalanki skolininkui, nes jis priverstinai praranda dalį arba visą savo užstatą. Skolintojams likvidacija naudinga tuo, kad užtikrina paskolos grąžinimą net ir tuomet, kai skolininkas pats to padaryti nesugeba. Todėl likvidacijos procesas yra naudingas likviduotojams ir skolintojams, tačiau nuostolinga skolininkui.

Jeigu valiuta, kuria reikia grąžinti skolą, yra nestabili iš likvidatoriaus perspektyvos, galima konvertuoti turima kapitalą iš stabilios valiutos į skolinamąją valiutą. Panašiai, užstato valiutą galima konvertuoti į stabilų piniginį vienetą per tam tikras valiutų keitimo operacijas kitose rinkose. Tokiu atveju, arbitražo sistema tampa itin vertinga, nes ji gali automatizuoti valiutos konversijos procesą ir rasti efektyviausią būdą konvertuoti iš valiutos A į valiutą B, tuo pačiu užtikrinant, kad likvidatorius gaus maksimalų pelną.

Taigi, jeigu turime algoritmą, kuris sugeba aptikti arbitražo galimybes iš tam tikros rinkų aibės, šią aibę galima papildyti likvidacijos rinkomis, leidžiant likvidatoriams bandyti pasipelnyti iš likvidacijų su minimalia ar net be rizikos. Tačiau likvidatoriai gali susidurti su rizikomis -- pavyzdžiui, būti aplenktiems kitų likvidatorių arba mokėti mokesčius už bandymus atlikti likvidaciją be sėkmingo likvidavimo dėl klaidos ar būsenos pasikeitimo, kuri nebeleistų atlikti valiutų konvertavimo. Šios rizikos yra aktualios ir paprastesniems arbitražo scenarijams tarp rinkų.

\section{Venus protokolo likvidavimo mechanizmas}
\label{sec:venus_mechanizmas}

Šiame skyriuje siekiama išsamiai išanalizuoti \textit{Venus} protokolo likvidavimo mechanizmą \cite{venusprotocolcode}, suprasti jo veikimo principus bei išskirti pagrindinius parametrus, lemiančius likvidacijos eigą. Nors pasirinktas konkretus protokolas, analizė nėra skirta tik jam – šiuo tyrimu taip pat siekiama įvertinti, kiek \textit{Venus} modelis yra atstovaujantis platesnei kriptovaliutų paskolų platformų ekosistemai. Lyginant su kitais protokolais, bus siekiama pagrįsti, kad tolimesni šiame darbe siūlomi algoritmai gali būti taikomi ir kitose panašaus tipo decentralizuotose skolinimo sistemose.

\subsection{Sąvokos}
\begin{itemize}
  \item \textbf{Pozicija (angl. \textit{position})}: rinkinys kelių užstatų ir paskolų, suskirstytų pagal valiutą, kuriuos valdo vienas naudotojas.
  \item \textbf{Užstato koeficientas / Paskolos ir užstato santykis (angl. \textit{collateral factor, CF} / \textit{loan to value, LTV})}: procentinė dalis, kuria užstato vertė įtraukiama į skolinimosi pajėgumą. Kiekviena valiuta turi atskirą vertę (paprastai nuo 60\% iki 90\%).
  \item \textbf{Skolinimosi pajėgumas (angl. \textit{borrowing capacity, BC})}: apibrėžia bendrą vertę, kurią skolininkas gali pasiskolinti, atsižvelgiant į jo užstato sumą.
    \begin{equation}
        BC = \sum_{i} \bigl(\text{Užstato vertė}_{i} \times CF_{i}\bigr), \quad \text{kur } i \text{ žymi } i\text{-ąją valiutą pozicijoje}
        \label{eq:borrowing_capacity}
    \end{equation}
  \item \textbf{Sveikumo koeficientas (angl. \textit{health factor, HF})}: matuoja pozicijos būklę, apibrėžiamą kaip skolinimosi pajėgumo ir esamų skolų santykį. Jeigu sveikumo koeficientas mažesnis negu 1, skolininkas tampa likviduojamas.
    \begin{equation}
        HF = \frac{BC}{\sum_{i} \text{Skolos vertė}_{i}}
        \label{eq:health_factor}
    \end{equation}
  \item \textbf{Likvidavimo slenkstis (angl. \textit{liquidation threshold})}: riba, nuo kurios skolininkas tampa likviduojamas.
  \item \textbf{Uždarymo riba (angl. \textit{close factor})}: maksimali skolos dalis, kuri leidžiama būti grąžinta vienos likvidacijos metu. Vertė yra bendra visam protokolui, mūsų atveju 50\%.
  \item \textbf{Likvidacijos paskata}: priedas arba nuolaida, kurią likvidatorius gali gauti, likviduodamas užstatą.
  Šis skirtumas skatina likvidatorius veikti nedelsiant, kai paskola peržengia likvidavimo slenkstį. \textit{Venus} atveju tai yra 10\% gražinamos valiutos vertės.
\end{itemize}

\subsection{Likvidacijos ypatumai}
\label{sec:likivdacijos_ypatumai}

Skolininkui tapus likviduojamu, yra daugybė pasirinkimų, kaip vykdyti jo pozicijos likvidavimą. Likvidavimo proceso metu būtina pasirinkti konkrečią užstato ir paskolos valiutų porą, kurioje paskolos valiuta yra grąžinama, o užstato valiuta paimama iš skolininko. Taip pat, nustatomas grąžintinos sumos dydis. Svarbu pabrėžti, jog atlikus nedidelę likvidaciją, skolininkas gali išlikti likviduojamas, tai reiškia, kad likvidavimo procesas gali būti vykdomas kelis kartus.

Svarbu pažymėti, kad likvidavimo metu protokolas leidžia likviduoti didesnį valiutos kiekį, nei būtina skolininko pozicijai subalansuoti. Tarkime:
\begin{itemize}
  \item skolininkas yra pateikęs \$2000 vertės BTC (bitkoino) valiutą kaip užstatą;
  \item BTC valiutos užstato koeficientas (\textit{CF}) yra 80\%;
  \item skolininkas yra pasiskolinęs 1650 USDT valiutos, kuri atitinka JAV dolerį.
\end{itemize}

Toliau \ref{tab:likvidacijos_pav} lentelėje matyti, kaip šios vertės kinta po skirtingo dydžio likvidacijų.
Pirmuoju variantu grąžinama \$417, po kurios skolininko sveikumo koeficientas pakyla virš 1, todėl
skolininko pozicija nustoja būti likviduojama.
Antruoju variantu grąžinama maksimali suma, kiek leidžiama, kai uždarymo riba lygi 50\%.
Šiuo atveju pozicija taip pat tampa „sveika“, tačiau skolininkas patiria didesnį nuostolį, o likvidatorius – didesnį pelną.

\begin{table}[h!]
  \centering
  \caption{Pavyzdinė užstato, skolos ir sveikumo koeficiento kaita po likvidacijos}
  \begin{tabular}{lccccc}
  \hline
  \textbf{Būsena} 
  & \textbf{Užstato vertė}
  & \textbf{Skolos vertė}
  & \textbf{BC}
  & \textbf{HF}
  & \makecell{\textbf{Pelnas likvidatoriui /}\\ \textbf{nuostolis skolininkui}} \\ 
  \hline
  Pradinė                
  & 2000      
  & 1650      
  & 1600      
  & 0,97    
  & --         \\
  
  Po \$417 likvidacijos  
  & 1541,3    
  & 1233      
  & 1233,04   
  & 1,00003   
  & 41,7       \\
  
  Po \$825 likvidacijos  
  & 1092,5    
  & 825       
  & 874       
  & 1,06      
  & 82,5       \\
  \hline
  \end{tabular}
  \label{tab:likvidacijos_pav}
  \end{table}

Likvidatoriams yra naudinga grąžinti kuo didesnę skolą už skolininką, nes už grąžintą paskolos valiutą atlyginama tokios pačios vertės užstatu, be to, papildomai suteikiama 10\% užstato kaip paskata. Likvidatorių procesą riboja du pagrindiniai dalykai: skolininko užstato dydis ir uždarymo riba.

Jeigu likvidatorius siekia vykdyti arbitražą, jis taip pat turi įvertinti kainos pokytį, kuris įvyks keičiant valiutas kitose rinkose. Dideli valiutų keitimai lemia didesnius kainos svyravimus nepalankia kryptimi. Todėl likvidatoriaus veiksmus netiesiogiai riboja ir kitų rinkų likvidumas. Jei valiutos likvidumas yra mažas, orakulo nurodyta kaina gali reikšmingai skirtis nuo realios rinkos kainos. Tokiais atvejais gali susidaryti situacija, kai net ir turėdamas galimybę atlikti didelės vertės (milijonų dolerių) likvidaciją, likvidatorius jos nevykdys, nes užstatas būtų realizuojamas gerokai žemesne kaina nei ta, kurią tuo metu rodo orakulas. 
% Šiame darbe dėl paprastumo daroma prielaida, kad gražinamą valiutą arba užstatą galima gauti ar iškeista orakulo kainomis. PAAISKINTI KODEL TAIP MUMS SAUGU.

\subsection{Panašumai su kitais protokolais}
\textcolor{red}{
Parodome, kad panašūs dalykai egzistuoja ir Aave’e, ir kituose didžiuosiuose protokoluose, tokiuose kaip Compound.
}

\section{Skirtingos likvidacijų strategijos}
\label{sec:liq_strategijos}

Šiame skyriuje pateikiami skirtingi likvidavimo metodai. Pirmiausia aprašomi paprasti, intuityvūs metodai, kuriuos nesudėtinga sugalvoti. Vėliau apžvelgiama esama literatūra, siūlanti efektyvius likvidavimo būdus. Taip pat pristatomi autoriaus pasiūlyti patobulinimai, atsižvelgiant į \textit{Venus} protokolo ypatumus. Galiausiai analizuojamos strategijos, kurios taikomos likviduojant keletą skirtingų valiutų porų.

\subsection{Naivus algoritmas}
\label{sec:largest_borrow}

Paprasta likvidavimo algoritmo idėja gali būti tokia: turint likviduojamo skolininko poziciją, pasirenkama paskolos valiuta, kuriai skolininkas turi didžiausią įsipareigojimą, ir užstato valiuta, kurios vertė skolininko pozicijoje yra didžiausia. Tuomet stengiamasi grąžinti tiek, kiek leidžia uždarymo riba (paprastai – pusę pasiskolintos valiutos vertės), arba tiek, kad būtų išnaudotas visas skolininko užstatas pasirinkta valiuta. Kadangi praktikoje dažniau pasiekiama uždarymo riba, o ne užstato apribojimas, šiai strategijai suteikiame pavadinimą – \textbf{didžiausia skola}.

Ši strategija literatūroje dar kartais vadinama \textbf{iki uždarymo ribos} (angl. \textit{up to close factor}) \cite{Emp}. Šiame darbe abu terminai vartojami skirtinguose kontekstuose: kai galime laisvai pasirinkti skolos ir užstato valiutas, strategija vadinama \textbf{didžiausia skola}, o kai valiutų pora jau yra fiksuota, vartojamas pavadinimas \textbf{iki uždarymo ribos}.

Svarbu pažymėti, kad skolininko užstato vertė yra ribojama išgryninimo galimybėmis – paskolų platformose likvidatorius užstatą gauna apgaubta forma (angl. \textit{wrapped}), kurią reikia papildomai konvertuoti į grynąją (angl. \textit{underlying}) valiutą. Gali kilti situacijų, kai grynoji užstato forma tuo metu yra nepasiekiama, nes visa valiuta yra paskolinta kitiems skolininkams. Dėl to ne visada įmanoma iš karto atsiimti visą užstato vertę. Toliau šiame darbe, kalbant apie skolininko užstato vertę, bus daroma prielaida, kad kalbama apie skolininko užstato dalį, kurią galima tuo metu išsigryninti.

\ref{tab:naivus_pav} lentelėje pateikti trys scenarijai su skolininko pozicijomis, kuriose, tarkime, visi skolininkai yra likviduojami:
\begin{enumerate}
    \item \textbf{Pirmas scenarijus:} renkamės ETH valiutą kaip grąžinamą, nes taip galima grąžinti didžiausią vertę (\$500). Grąžinant šią sumą iš skolininko užstato galima paimti \$550 vertės turto. Kadangi BNB yra tik \$500, to nepakanka, todėl renkamės valiutą su didžiausiu užstatu – BTC.
    
    \item \textbf{Antras scenarijus:} renkamės USDT kaip grąžinamą valiutą, o ETH – kaip užstatą, nes jo vertė didžiausia. Tiesa, šiuo atveju pakaktų ir AVAX užstato, nes grąžindami didžiausią leistiną sumą (\$1000) galime paimti \$1100 vertės turto. Jei AVAX valiutą būtų lengviau parduoti rinkoje, likvidatorius galėtų rinktis ją.

    \item \textbf{Trečias scenarijus:} iš kelių užstato variantų renkamės ETH, nes jo vertė didžiausia, o grąžinama valiuta bus USDC (taip pat dėl didžiausios sumos). Iš tikrųjų bet kuri grąžinama valiuta leistų padengti tokią sumą, jog būtų galima paimti visą ETH užstatą, nes tereikia grąžinti $\frac{\$500}{1,1} \approx \$454,54$ skolos.
\end{enumerate}

\begin{table}[h!]
  \centering
  \caption{Pavyzdiniai skolininko pozicijų duomenys su įvairiomis skolomis ir užstatais}
  \begin{tabular}{lcc}
  \hline
  \textbf{Pozicija} 
   & \textbf{Skolos} 
   & \textbf{Užstatai} \\
  \hline
  1 & 
  \begin{tabular}[c]{@{}l@{}}ETH: \$1000 \\ DAI: \$500 \\ USDC: \$300\end{tabular} 
   & \begin{tabular}[c]{@{}l@{}}BTC: \$2000 \\ BNB: \$500\end{tabular} \\
  \hline
  2 &
  \begin{tabular}[c]{@{}l@{}}USDT: \$2000 \\ DAI: \$400\end{tabular}
   & \begin{tabular}[c]{@{}l@{}}ETH: \$3000 \\ AVAX: \$1500\end{tabular} \\
  \hline
  3 &
  \begin{tabular}[c]{@{}l@{}}USDC: \$1500 \\ WBTC: \$800\end{tabular}
   & \begin{tabular}[c]{@{}l@{}}ETH: \$500 \\ SOL: \$400 \\ AVAX: \$400 \\ BNB: \$400\end{tabular} \\
  \hline
  \end{tabular}
  \label{tab:naivus_pav}
  \end{table}

\subsection{Esami tyrimai}
\label{sec:esami_tyrimai}

Šioje srityje jau egzistuoja mokslinių darbų, siūlančių algoritmus likvidavimo problemoms spręsti. Pavyzdžiui, \cite{Emp} pristato algoritmą \textit{Optimal Fixed Spread Liquidation Strategy}, kuris padalina likvidavimo procesą į dvi mažesnes likvidacijas, kurias kartu sudėjus, leidžia likviduoti didesnį kiekį nei būtų galima padaryti tik su vienu dideliu likvidacijos iškvietimu. Pirmoji likvidacija yra maksimaliai didelė, taip, kad po jos skolininkas vis dar lieka likviduojamas, reiškia, jog sveikumo koeficientas (\textit{HF}) yra kiek įmanoma didesnis, bet mažesnis nei 1. Antroji likvidacija užbaigia procesą, likviduojant maksimalų kiekį, kurį leidžia protokolo taisyklės, vienu likvidacijos iškvietimu, kas paprastai reiškia 50\% likusios paskolos dydžio. Ši strategija veikia tik tada, kai yra iš anksto nurodyta skolos ir užstato valiutų pora – ji pati nenusako, kokią porą pasirinkti, o siekia maksimaliai padidinti likvidacijos dydį konkrečiai pateiktoje poroje.

\textit{Venus} protokolui \textit{Optimal Fixed Spread Liquidation Strategy} algoritmas nėra tinkamas, nes grąžinant pagrindinio tinklo valiutą (šiuo atveju BNB) pasikeičia kursas, pagal kurį apgaubtas užstatas (angl. \textit{wrapped}) konvertuojamas į pagrindinę valiutą. Kurso perskaičiavimas įvyksta prieš pat nustatant skolininko pozicijos sveikumo koeficientą ($HF$). Dėl to didelės grąžinamos sumos gali reikšmingai paveikti skolininko likvidumo būklę, sumažindamos likvidacijos ribą arba netgi visiškai sustabdydamos likvidavimo procesą. Pagrindinė problema su algoritmu yra ta, kad pirmoji likvidacija, nors formaliai palieka skolininką likviduojamą, tam tikrais atvejais leidžia likviduoti tik labai mažas sumas antrojoje likvidacijoje.

Tam tikrais atvejais, kai skolininko įsiskolinimas yra itin didelis, gali būti vykdomos daugiau nei dvi likvidacijos, kiekvieną kartą likviduojant iki uždarymo ribos. Tačiau minėtas algoritmas apsiriboja tik dviem likvidacijų iškvietimais.

\subsection{Pasiūlyta \textit{Optimal Fixed Spread Liquidation Strategy} modifikacija}

Siūlome koreguoti \textit{Optimal Fixed Spread Liquidation Strategy}. Visų pirma, neapsiribosime dviem likvidacijos iškvietimais. Proceso pradžioje sukame ciklą, kurio metu kiekvieno iteracijos pradžioje apskaičiuojame $\text{B}_{\text{uždarymo}}$ ir $\text{B}_{\text{užstato}}$, kur:

\begin{itemize}
\item $\text{B}_{\text{uždarymo}}$ – maksimalus grąžinamos valiutos kiekis, leidžiamas likviduoti vienu kartu pagal uždarymo ribos taisyklę.
\item $\text{B}_{\text{užstato}}$ – grąžinamos valiutos kiekis, reikalingas visiškai atsiimti skolininko užstatą pasirinkta valiuta.
\end{itemize}

Jei $\text{B}_{\text{užstato}} \leq \text{B}_{\text{uždarymo}}$, vadinasi, kita likvidacija bus paskutinė, ir grąžiname $\text{B}_{\text{užstato}}$ pasiskolintos valiutos kiekį. Šios logikos privalumas yra tas, kad jei pirmosios likvidacijos dydis yra ribojamas užstato trūkumo, procesas gali būti baigtas per vieną likvidaciją. Tai leidžia maksimaliai efektyviai išnaudoti galimybes, sumažinant nereikalingų skaičiavimų, už kuriuos reikia sumokėti pagrindine valiuta, kiekį.

Kitu atveju ($\text{B}_{\text{užstato}} > \text{B}_{\text{uždarymo}}$), ieškome didžiausio $\text{B}_{i}$ tokio, kad po likvidacijos vis dar būtų galima likviduoti bent $\text{B}_{\text{užstato}} - \text{B}_{i}$. Jei rastas $\text{B}_{i} > 0$, atlikus likvidaciją grįžtame į ciklo pradžią. Jei $\text{B}_{i} = 0$, reiškia, bet koks likvidacijos dydis pavers skolininką nelikviduojamu. Tokiu atveju likviduojama $\text{B}_{\text{uždarymo}}$ ir į ciklo pradžią nebegrįžtama.

Galima įsivaizduoti, kad \ref{sec:esami_tyrimai} skyriuje aprašytas algoritmas taip pat ieško didžiausio $\text{B}_{i}$ pirmajai likvidacijai, užtikrinančio, jog po jos skolininkas išliks likviduojamas. Tačiau mūsų algoritmas reikalauja ne tik, kad skolininkas išliktų likviduojamas, bet ir kad būtų galima likviduoti reikšmingą kiekį valiutos. Šis papildomas reikalavimas yra ypač aktualus tais atvejais, kai grąžinama valiuta yra pagrindinė blokų grandinės valiuta – BNB.

Tais atvejais, kai skolininkas yra stipriai įsiskolinęs, siūlomas algoritmas atliks seką likvidacijų, kurių kiekviena bus dvigubai mažesnė už ankstesnę. Tai tęsis tol, kol bus grąžinta visa skola arba išnaudotas visas užstatas. Dėl šios savybės šią modifikuotą strategiją vadiname \textbf{pilno išeikvojimo} strategija (angl. \textit{drain strategy}).

\subsection{Pilnas išeikvojimas vienodoms valiutoms}

Ši strategija yra \textbf{pilno išeikvojimo} strategijos specializacija, taikoma tik tiems atvejams, kai skolos ir užstato valiutos sutampa. Tokiose situacijose atliekamos pilnos išeikvojimo likvidacijos visoms valiutų poroms, kur skolos ir užstato valiuta yra identiška. Prioritetas teikiamas toms valiutoms, kurių reikšmė $\min(\text{skolos vertė}, \text{užstato vertė} / \text{likvidacijos paskata})$ yra didžiausia, ir einama žemyn, kol skolininkas nebėra likviduojamas. Šios strategijos privalumai – sumažinta valiutų konvertavimo rizika, išvengiama papildomų mokesčių, susijusių su keitimu, bei sumažinama likvidumo trūkumo tikimybė kitose rinkose, nes visa likvidacija vyksta vienoje valiutų poroje. Pelnas tokiu atveju taip pat išmokamas ta pačia valiuta.

% Norint pritaikyti šią strategiją, būtina, kad skolininko pozicijoje būtų ta pati valiuta tiek skolos, tiek užstato pusėje. Tokios pozicijos nėra dažnos, tačiau jos visgi pasitaiko praktikoje. Tokia struktūra gali atsirasti dėl specifinių strategijų. Viena jų \textbf{paskatų rinkimas (angl. incentive reward farming).} \textit{Venus} protokole tiek skolintojai, tiek skolininkai gali gauti paskatas tam tikromis valiutomis, jei tuo metu konkrečiai valiutai yra aktyvi paskatų programa \cite{venusrewards}. Pateikus užstatą USDT ir tuo pačiu pasiskolinus USDT, naudotojas gali gauti paskatas abiejose pozicijos pusėse. Pavyzdžiui, jei užstato metinė grąža siekia 5\%, o skolos palūkanos – 9\%, tai iš pirmo žvilgsnio reikštų 4\% nuostolį, jei pasiskolinta suma investuojama be papildomos grąžos. Tačiau jei tuo metu paskatos siekia po 6\% tiek už užstatą, tiek už skolą, bendras rezultatas tampa teigiamas: 5\% (užstato palūkanos) + 6\% (užstato paskata) + 6\% (skolos paskata) − 9\% (skolos palūkanos) = +8\% metinė grąža. Tokiu atveju pasiskolintus USDT galima pakartotinai naudoti kaip užstatą, didinant bendrą pozicijos dydį ir gaunant papildomą pelną nuo pradinių investuotų lėšų. Kiti protokolai kaip AAVA taip pat turi paskatų sistemą \cite{aaveincentives}.

Norint pritaikyti šią strategiją, būtina, kad skolininko pozicijoje būtų ta pati valiuta tiek skolos, tiek užstato pusėje. Nors tokios pozicijos pasitaiko retai, jos gali atsirasti dėl tam tikrų paskatų struktūrų. \textit{Venus} protokole tiek skolintojai, tiek skolininkai gali gauti papildomas paskatas konkrečiomis valiutomis, jei tai numatyta aktyvioje paskatų programoje \cite{venusrewards}. Pavyzdžiui, jei naudotojas užstato USDT ir kartu pasiskolina USDT, jis gali gauti paskatas abejose pozicijos pusėse. Tarkime, kad užstato metinė grąža yra 5\%, o skolos palūkanos – 9\%. Iš pirmo žvilgsnio tai reikštų 4\% nuostolį, jei pasiskolinta suma investuojama be papildomos grąžos. Tačiau jei paskatos siekia po 6\% tiek už užstatą, tiek už skolą, bendras rezultatas tampa teigiamas: 5\% (užstato palūkanos) + 6\% (užstato paskata) + 6\% (skolos paskata) − 9\% (skolos palūkanos) = +8\% metinė grąža. Tokiu atveju pasiskolinti USDT gali būti pakartotinai užstatomi, taip padidinant pozicijos dydį ir generuojant papildomą grąžą nuo pradinės investicijos. Panašias paskatų sistemas naudoja ir kiti protokolai, tokie kaip \textit{Aave} \cite{aaveincentives}.

% \begin{itemize}
%     \item \textbf{Paskatų rinkimas (angl. incentive reward farming).} \textit{Venus} protokole tiek skolintojai, tiek skolininkai gali gauti paskatas tam tikromis valiutomis, jei tuo metu konkrečiai valiutai yra aktyvi paskatų programa \cite{venusrewards}. Pateikus užstatą USDT ir tuo pačiu pasiskolinus USDT, naudotojas gali gauti paskatas abiejose pozicijos pusėse. Pavyzdžiui, jei užstato metinė grąža siekia 5\%, o skolos palūkanos – 9\%, tai iš pirmo žvilgsnio reikštų 4\% nuostolį, jei pasiskolinta suma investuojama be papildomos grąžos. Tačiau jei tuo metu paskatos siekia po 6\% tiek už užstatą, tiek už skolą, bendras rezultatas tampa teigiamas: 5\% (užstato palūkanos) + 6\% (užstato paskata) + 6\% (skolos paskata) − 9\% (skolos palūkanos) = +8\% metinė grąža. Tokiu atveju pasiskolintus USDT galima pakartotinai naudoti kaip užstatą, didinant bendrą pozicijos dydį ir gaunant papildomą pelną nuo pradinių investuotų lėšų. Kiti protokolai kaip AAVA taip pat turi paskatų sistemą \cite{aaveincentives}.
    
%     % \item \textbf{Leverege liquid stake yields} DeFi egzistuoja liquid stake valiutos kurios leidzia disponuoti investicijas kaip betkokia kita valiutą. Pavyzdziui \cite{AstherusStakedBNB} Astherus Staked BNB (ASBNB) works on a simple yet effective principle. When users deposit their BNB into the Astherus staking protocol, they receive an equivalent amount of ASBNB tokens. These ASBNB tokens represent both the original BNB deposit and the accruing staking rewards. The value of ASBNB gradually increases relative to BNB as staking rewards are earned and reflected in the token's exchange rate. Šį investavimo tipa galima suintensyvinti pasinaudojant paskolu platformomis. Apie tokias galimybes detaliau rašo \cite{LeverageStaking}. Strategija atrodo taip: kaip uzstast inesamas BNB valiuta

% \end{itemize}

% Papasakoti kad va yra bent dvi strategijos skatinancios skolintis ir uzstatyti tas pacias valiutas, todel gebejimas likviduoti tokias pozicijas yra aktualus.

\subsection{Nuo didžiausio užstato koeficiento}

Toliau šiame darbe yra siųloma strategija \textbf{nuo didžiausio užstato koeficiento}, kuri maksimizuoja bendrą likvidacijos dydį su fiksuota paskolos valiuta ir nefiksuota užstato valiutą. Strategija apibrežia kurią užstato valiuta likviduoti tam tikru momentu priklausomai nuo skolininko pozicijos.

Grįžtant prie sveikumo koeficiento formulės \ref{eq:health_factor}, galima pastebėti, kad skolinimosi pajėgumas priklauso nuo užstato koeficientų – kuo didesnis užstato koeficientas (CF), tuo labiau atitinkamos valiutos vertė prisideda prie skolinimosi pajėgumo. Likvidatorius gali turėti galimybę paveikti, kaip greitai mažėja skolinimosi pajėgumas, priklausomai nuo to, kuri užstato valiuta pasirenkama likviduoti. 

Likviduojant užstatą su aukštu užstato koeficientu, skolinimosi pajėgumas sumažėja labiau nei likviduojant valiutą su mažu koeficientu, nes prarandama labiau „vertinama“ užstato dalis. Tokiu būdu skolininko sveikumo koeficientas (HF) gali išlikti žemiau 1 (t.y. išlikti likviduojamu) ilgiau , net ir gražinant tą pačią paskolos sumą – tai leidžia atlikti visumoje didesnę likvidaciją ir likvidatoriui daugiau uždirbti.

Tarkime, kad:
\begin{itemize}
    \item $R$ – grąžinamos paskolos vertė (angl. \textit{return});
    \item $I$ – likvidacijos paskata (angl. \textit{incentive}), išreiškianti, kiek užstato proporcingai atimama nuo grąžinamos skolos vertės;
    \item $CF$ – užstato koeficientas (angl. \textit{collateral factor});
    \item $LV$ – bendra skolininko skolų vertė (angl. \textit{loan value}) ($\sum_{i} \text{Skolos vertė}_{i}$).
\end{itemize}

Tuomet sveikumo koeficientas po likvidacijos gali būti apskaičiuotas taip:
\begin{equation}
    HF_{\text{po}} = \frac{BC_{\text{prieš}} - R \cdot I \cdot CF}{LV_{\text{prieš}} - R}
    \label{eq:health_factor_delta}
\end{equation}

Kad skolininkas po likvidacijos išliktų likviduojamas, turi būti tenkinama nelygybė:
\begin{equation}
    {BC_{\text{prieš}} - R \cdot I \cdot CF} < {LV_{\text{prieš}} - R}
    \label{eq:nesveikumo_inequalinty}
\end{equation}

Perkėlus narius į kitą pusę, gauname tokią sąlygą:
\begin{equation}
    BC_{\text{prieš}} < LV_{\text{prieš}} + R \cdot (I \cdot CF - 1)
    \label{eq:nesveikumo_inequalinty_2}
\end{equation}

Išraiškai $(I \cdot CF - 1)$ suteikiame pavadinimą \textbf{gydymo koeficientas}, nes ji nusako, kaip greitai nelygybė artėja prie ribos didinant grąžinamos paskolos dydį $R$. \ref{tab:cf_scenarijai} lentelėje pateikti keli pavyzdiniai scenarijai su skirtingais užstato koeficientais.

% Užstato koeficiento ($CF$) įtaka gydymo koeficientui
\begin{table}[H]
    \centering
    \caption{Užstato koeficiento ($CF$) įtaka gydymo koeficientui}
    \begin{tabular}{|c|c|}
        \hline
        \textbf{Užstato koeficientas ($CF$)} & \textbf{Gydymo koeficientas ($I \cdot CF - 1$), kai $I=1.1$} \\
        \hline
        0.00 & -1.00 \\
        0.60 & -0.34 \\
        $\frac{1}{1.1} \approx 0.91$ & 0.00 \\
        1.00 & +0.10 \\
        \hline
    \end{tabular}
    \label{tab:cf_scenarijai}
\end{table}

Kuomet likviduojamo skolininko pozicijoje yra užstatas, kurio užstato koeficientas $CF$ tenkina sąlygą $CF \geq \frac{1}{I}$, gydymo koeficientas tampa neneigiamas. Tokiu atveju nelygybė \ref{eq:nesveikumo_inequalinty_2} visada bus tenkinama, nepriklausomai nuo $R$ vertės, ir skolininkas išliks likviduojamas tol, kol bus likęs užstatas jo pozicijoje. Tai yra itin nepalanku skolininkui, nes tokiu atveju uždarymo riba praranda savo funkciją – apriboti likvidacijos žalą skolininkui. Tokia situacija laikoma nepageidaujama, todėl paskolų platformos įprastai ir nenaudoja tokių aukštų užstato koeficientų, kad būtų išvengta galimybės likviduoti visą poziciją be apribojimo.  Venus protokolas šiuo atveju net turi į išmanųjį kontraktą įrašytą apribojimą, neleidžiantį nustatyti užstato koeficiento didesnio nei 90\% \cite{venusmaxcf}.

Priešingu atveju, kai $CF < \frac{1}{I}$, egzistuoja riba, prie kurios didėjant $R$, nelygybė \ref{eq:nesveikumo_inequalinty_2} nustoja galioti. Tokiu atveju galima rasti tokią $R$ reikšmę, kad po likvidacijos skolininko sveikumo koeficientas taptų artimas 1 iš apačios. 
Išreiškiame $R_{\text{max}}$:

\begin{equation}
    R_{\text{max}} = \frac{LV_{\text{prieš}} - BC_{\text{prieš}}}{1 - I \cdot CF} - \varepsilon
    \label{eq:max_R}
\end{equation}
čia $\varepsilon$ – mažiausias įmanomas nedalomas grąžinamos paskolos vienetas, kuriuo galima sumažinti $R$, kad sveikumo koeficientas būtų artimas 1 iš apačios.

Iš \ref{eq:max_R} formulės matyti, kad kuo didesnis užstato koeficientas ($CF$), tuo didesnė gali būti $R_{\text{max}}$, vadinasi, likvidatorius gali grąžinti didesnę skolos sumą ir gauti didesnį atlygį. Todėl galima daryti išvadą, kad likvidatoriui yra efektyvu, kol sveikumo koeficientas išlieka mažesnis nei 1, pasirinkti tokias skolos ir užstato valiutas, kurių užstato valiuta turi didžiausią $CF$.
% Jei kelių užstato valiutų $CF$ yra vienodi, pirmiausia pasirenkamos tos, kurių vertė yra mažiausia. Šio pasirinkimo nauda bus paaiškinta vėliau.

Kai skolininko sveikumo koeficientas yra arti 1, reikšmė $R_{\text{max}}$ tampa artima nuliui, todėl tolesni likvidacijos kvietimai tampa ekonomiškai neefektyvūs. Tokiu atveju strategija nurodo pereiti prie „didžiausios skolos“ strategijos (\ref{sec:largest_borrow}), pasirenkant užstato valiutą, kurios vertė skolininko pozicijoje yra didžiausia. Tokiu būdu galima užtikrinti maksimalų pelną, likviduojant didesnę skolos dalį per vieną iškvietimą, nebepriklausant nuo gydymo koeficiento ribojimų.

% Strategijos pabaigoje siekiama atlikti kuo didesnę paskutinę likvidaciją. Todėl, jei įmanoma, didžiausią vieno kvietimo likvidaciją naudinga pasilikti pabaigai. Kadangi paskolos valiuta šioje strategijoje yra fiksuota, tai reiškia, jog verta pasilikti užstato valiutą, kurios vertė yra didžiausia, neatsižvelgiant į jos užstato koeficientą, kadangi paskutinėje likvidacijoje šis koeficientas nebėra aktualus.

Taigi grįžtant į pradžią – strategija prasideda nuo visų skolininko užstato valiutų surinkimo ir jų surikiavimo mažėjančia tvarka pagal užstato koeficientą. Jei kelių valiutų koeficientai yra vienodi, papildomai jos rikiuojamos pagal užstato vertę didėjančia tvarka. Iš surikiuoto sąrašo paeiliui pasirenkama pirmoji (t. y. geriausiai reitinguota) užstato valiuta ir su ja atliekama likvidacija. Atliekama kuo didesnė galima likvidacija, siekiant maksimaliai sumažinti skolininko skolinimosi pajegumą ($BC$), kartu išlaikant skolininką likviduojamu ($HF < 1$). Kiekviename cikle tikrinama, ar liko tik viena užstato valiuta – jei taip, su ja taikoma pilno išeikvojimo strategija. Jei dar liko kelios valiutos, įvertinama, ar sąraše yra bent viena valiuta, kurios vertė didesnė nei šiuo metu nagrinėjamos. Jei tokia yra, reiškia dabartinę valiutą galima naudoti dėl jos aukšto užstato koeficiento, o kitą, vertingesnę – pasilikti pabaigai. Jei dabartinė valiuta turi didžiausią vertę, bandoma ją išeikvoti taikant pilno išeikvojimo strategiją ir nustatoma, ar apribojimas kilo dėl užstato ar paskolos dydžio. Jei apribojimas yra paskolos pusėje – reiškia, likvidacija išnaudota maksimaliai ir strategija baigiama. Jei apribojimas kilo dėl užstato trūkumo, strategija tolimesnių veiksmų šioje vietoje neapibrėžia.

Turint užstato valiutą su didžiausia verte, iškyla neaiškumas – ar ją verta naudoti ankstesniuose žingsniuose dėl jos aukšto užstato koeficiento, kuris leidžia ilgiau išlaikyti skolininką likviduojamą, ar geriau pasilikti šią valiutą paskutinei likvidacijai, kai siekiama maksimalaus grąžinamos paskolos kiekio pagal „didžiausios skolos“ strategiją. Šio pasirinkimo optimizavimas priklauso nuo konkrečių pozicijos parametrų ir yra paliekamas kitiems tyrimams.

\subsection{Nuo mažiausio užstato koeficiento}

Ši strategija yra analogiška \textbf{nuo didžiausio užstato koeficiento} strategijai, tačiau skirtumas slypi tvarkoje: užstato valiutos pradiniame sąraše rikiuojamos didėjančia tvarka pagal užstato koeficientą. Jei kelių valiutų koeficientai sutampa, jos toliau rikiuojamos pagal užstato vertę didėjančia tvarka.

Šios strategijos tikslas – išsiaiškinti, ar užstato valiutų tvarka turi reikšmingos įtakos likvidatoriaus galutiniam pelnui. Ji įtraukta tyrimo tikslais kaip kontrolinė versija, leidžianti palyginti rezultatų skirtumus su strategija, kurioje tvarka yra priešinga.

\subsection{Strategijų apibendrinimas}
Visoms šiame darbe nagrinėjamoms strategijoms taikomas tas pats bendras apribojimas – likvidacija nėra vykdoma, jei jos pelnas yra mažesnis nei kuro sąnaudos, reikalingos likvidacijos kvietimui. Tokiu būdu užtikrinama, kad kiekvienas likvidacijos bandymas būtų ekonomiškai pagrįstas ir neturėtų neigiamo pelno.

Svarbu priminti, kad kalbant apie skolininko užstato vertę, šiame darbe visur daroma prielaida, jog turima omenyje tik tą užstato dalį, kuri tuo metu gali būti išsigryninta (angl. \textit{underlying}), o ne apgaubtoje (angl. \textit{wrapped}) formoje. Toks supaprastinimas leidžia išvengti situacijų, kai skola niekada negrąžinama, o užstato konvertavimas tampa neįmanomas dėl protokolo likvidumo stokos. Be to, toks apribojimas yra ypač svarbus vykdant arbitražą – siekiant pasinaudoti pelno galimybe, būtina kuo greičiau išsigryninti gautą užstatą ir, jei reikia, atlikti papildomus valiutų keitimus tam, kad būtų grįžta prie arbitražo vykdytojo pelno valiutos.

Kai kurios strategijos papildomai apibrėžia, kaip turi būti parenkamos skolos ir/arba užstato valiutos, tuo tarpu kitos daro prielaidą, kad šie pasirinkimai jau yra nustatyti. Taip pat skiriasi strategijų apimtis – dalis jų apribotos viena likvidacija, o kitos gali nurodyti kelių likvidacijų seką.

Strategijų skirtumai glaustai pateikti \ref{tab:strategiju_klasifikacija} lentelėje.

\begin{table}[H]
    \centering
    \caption{Strategijų klasifikacija pagal skolos ir užstato valiutų fiksavimą ir kvietimų skaičių}
    \begin{tabular}{|>{\raggedright\arraybackslash}p{5.5cm}|c|c|c|}
      \hline
      \textbf{Strategija} & \makecell{\textbf{Skolos valiuta} \\ \textbf{fiksuota}} & \makecell{\textbf{Užstato valiuta} \\ \textbf{fiksuota}} & \makecell{\textbf{Galimas daugiau nei} \\ \textbf{vienas likvidacijos kvietimas}} \\
      \hline
      Iki uždarymo ribos & taip & taip & ne \\
      \hline
      Didžiausia skola   & ne & ne & ne \\
      \hline
      Pilnas išeikvojimas  & taip & taip & taip \\
      \hline
      Pilnas išeikvojimas vienodoms valiutoms  & ne & ne & taip \\
      \hline
      Nuo didžiausio/mažiausio užstato koeficiento & taip & ne & taip \\
      \hline
    \end{tabular}
    \label{tab:strategiju_klasifikacija}
\end{table}
  

\section{Strategijų tyrimas}
Tikslas – palyginti \ref{sec:liq_strategijos} skyriuje aprašytas likvidavimo strategijas pagal jų pelningumą likvidatoriaus atžvilgiu.

Atkartosime istorines \textit{Venus} protokolo \textit{Core} baseino likvidacijas ir joms priskirsime \textit{atkartoti} strategiją. Tai leidžia įvertinti faktinį tuo metu naudotą sprendimą ir palyginti jį su kitomis strategijomis. Strategijose, kuriose būtina nurodyti fiksuotą skolos ar užstato valiutą, bus naudojamos tos pačios valiutos, kokios buvo istorinėje likvidacijoje – taip užtikrinamos identiškos pradinės sąlygos. Taip pat kiekvienai simuliacijai bus taikoma ta pati kuro kaina, kurią savo tranzakcijoje nustatė originalus likvidacijos vykdytojas. Tokiu būdu strategiškai atkuriamos realios tuo metu galiojusios sąlygos, leidžiančios sąžiningai palyginti skirtingų strategijų rezultatus.

\subsection{Pelningumo skaičiavimas}

Darbe nagrinėjamos strategijos bus lyginamos pagal jų generuojamą pelną – kiekvienai strategijai apskaičiuosime grynąją naudą, atsižvelgiant į orakulo kainas bei kuro sąnaudas. Toliau pateikiamas pelningumo apibrėžimas ir formulė, kuria remsimės analizuodami skirtingų strategijų efektyvumą.

\[
\text{Pelnas} = \sum_{i} C_i \cdot P_{c_i} - \sum_{j} B_j \cdot P_{b_j} - Gas \cdot P_\text{bnb}
\]

\begin{itemize}
    \item $C_i$ – $i$-osios užstato valiutos kiekis, atgaunamas likvidatoriams;
    \item $P_{c_i}$ – $i$-osios užstato valiutos kaina orakule;
    \item $B_j$ – $j$-osios pasiskolintos valiutos kiekis, grąžinamas likvidacijos metu;
    \item $P_{b_j}$ – $j$-osios pasiskolintos valiutos kaina orakule;
    \item $Gas$ – bendras sunaudotas kuro kiekis;
    \item $P_\text{bnb}$ – BNB (pagrindinės grandinės valiutos) kaina.
\end{itemize}

\textit{Gas} sąnaudos – tai bendra sandorio vykdymo kaina, kuri gali būti suskirstyta į tris pagrindinius komponentus:
\begin{itemize}
    \item \textbf{Leidimo suteikimas (angl. \textit{approve})} – BSC blokų grandinėje, norint suteikti kitam adresui teisę pervesti valiutą iš savo adreso, būtina iškviesti atitinkamos valiutos kontrakto funkciją \texttt{approve}. Ji leidžia nurodytam adresui disponuoti konkrečiu valiutos kiekiu. \textit{Venus} protokole ši operacija reikalinga tik tuomet, kai grąžinama valiuta nėra pagrindinė tinklo valiuta – BNB.
  
    \item \textbf{Likvidacija (angl. \textit{liquidate})} – tai pati skolos padengimo operacija, kuri vykdoma iškviečiant \texttt{liquidateBorrow} funkciją. Šios funkcijos vykdymas sunaudoja kurą (gas), priklausomai nuo pozicijos dydžio ir pasirinktos valiutų poros.
  
    \item \textbf{Užstato išgryninimas (angl. \textit{redeem})} – po \texttt{liquidateBorrow} likvidatorius gauna „apgaubtą“ (angl. \textit{wrapped}) valiutą, kuri nėra tiesiogiai naudojama. Norint gauti tikrąją (angl. \textit{underlying}) valiutą, būtina ją iškeisti naudojant \texttt{redeem} funkciją. Ši funkcija taip pat reikalauja papildomo kuro kiekio. Svarbu pažymėti, kad išgryninimas gali būti neįmanomas, jei protokole tuo metu nėra pakankamai likvidumo – tai gali nutikti, jei visa ta valiuta yra paskolinta kitiems vartotojams.
\end{itemize}

\subsubsection{Pelno formulės trūkumas}

Nors ši pelningumo formulė turi stiprią koreliaciją su tikru pelnu, tai ne visada reiškia didžiausią arbitražo uždarbį. Kai orakulo kainos smarkiai skiriasi nuo kitų rinkų, gali būti pelningiau pasirinkti kitą valiutų porą – tiek skolos, tiek užstato pusėje. Net jei orakulo duomenimis pasirinkta valiuta atrodo mažiau vertinga, ją konvertuojant kitose rinkose gali būti pasiektas didesnis pelnas. Taip nutinka dėl to, kad skirtingos valiutų poros turi nevienodas konvertavimo sąlygas – vienos pasižymi geresniu likvidumu ar mažesniais mokesčiais.

Praktikoje kuro sąnaudos gali būti didesnės, nes prieš atliekant likvidaciją norima surinkti duomenis apie skolininko poziciją ir apskaičiuoti optimalią grąžintiną sumą. Be to, arbitražo atveju papildomas kuras sunaudojamas valiutų keitimui kitose rinkose. Šiame darbe minėti pasiruošimo skaičiavimai ir keitimai į analizę nėra įtraukti vertinant likvidavimo strategijų pelningumą, nes tam reikėtų gerokai platesnės analizės, į kurią būtų įtrauktos ir kitų rinkų kainos bei jų likvidumas konkrečiu laiko momentu.

\subsection{Duomenys}

Tyrimui atlikti buvo surinkti visi \textit{Venus} protokolo likvidacijų duomenys iš BSC tinklo iki 43,683,481 bloko (2024 m. lapkričio 3 d., 10:40:23 UTC). Iš viso buvo aptikta 63982 likvidacijos. Duomenų rinkimui buvo panaudoti \textit{quicknode.com} archyviniai serveriai.

\href{https://bscscan.com/tx/0x2434f5aee00e1135c66fb42203d506351ed2b6629e01af5daee37f652e4d67b8}{Pirmoji}
likvidacija 2020-11-26

\subsection{Strategijų realizacija}

Naudojant \textit{Forge} (blokų grandinėms skirtą testavimo karkasą, paremtą Ethereum), sukurta analizės aplinka, leidžianti simuliuoti pasirinktą likvidacijos strategiją konkrečiame bloke ir tranzakcijos pozicijoje.

Pačios strategijos buvo rašomos \textit{Solidity} programavimo kalba. Dėmesys sutelkiamas tik į likvidacijos dalį, todėl į analizę neįtraukiami pasiskolintos valiutos gavimas ir užstato valiutos pardavimas. Siekiant, kad likvidacijos būtų sėkmingos, simuliacijos pradžioje grąžinama valiuta fiktyviai priskiriama kontroliuojamai piniginei.

Grąžinamos sumos dydžiui nustatyti parašytos funkcijos, kurios iš blokų grandinės paima duomenis apie skolininką. Rezultatams patikrinti šios grąžinamos sumos papildomai testuojamos su padidinta verte, kad būtų galima įsitikinti, jog didesnė likvidacijos suma lemia nepageidaujamų situacijų (pvz., nebegalima toliau likviduoti). Nors grąžinamų sumų skaičiavimai ir papildomos simuliacijos naudoja blokų grandinės skaičiavimo kuro vienetus (angl. \textit{gas}), jie nėra įtraukiami į likvidacijos mokesčius, nes likvidacijos reikšmes galima apskaičiuoti kitoje (ne blokų grandinės)  aplinkoje ir tik tuomet įkelti jas į likvidaciją vykdančias tranzakcijas.

\subsection{Istorinių likvidacijų analizės pavyzdys}

\small
Šiame skyriuje nagrinėsime likvidaciją tranzakcijoje \\\href{https://app.blocksec.com/explorer/tx/bsc/0x8d286fa28b0eb4d4d4e1a8cdaf078190f207921f5d1a5f198de56f5995e2c606}{0x8d286fa28b0eb4d4d4e1a8cdaf078190f207921f5d1a5f198de56f5995e2c606}.

\begin{figure}[H]
  \centering
  \includegraphics[scale=0.3]{img/liquidation_example.png}
  \caption{Tranzakcijos valiutų pavedimai \cite{liqpvz}}
  \label{img:liquidation_example}
\end{figure}

\begin{enumerate}
  \item \textbf{Trumpalaikė paskola.}  
  (1) pavedimu likvidatorius iš valiutų keityklos gauna 1213110 BUSD-T, su sąlyga, kad toje pačioje tranzakcijoje grąžins kitą valiutą – BUSD. Tai yra pirmasis žingsnis akimirksnio apsikeitimo (angl. \textit{flash swap}).

  \item \textbf{Likvidacijos veiksmai.}  
  (2) ir (3) pavedimais likvidatorius persiunčią visą prieš tai gautą BUSD-T į vUSDT baseiną, taip dalinai grąžindamas 0x04d... skolininko skolą. Mainais iš skolininko sąskaitos gauna vETH (\textit{Venus} protokolo apgaubtas ETH).

  \item \textbf{vETH konvertavimas į įprastą ETH.}  
  (5) ir (6) pavedimais likvidatorius išsikeičia turimus vETH į paprastą ETH, iš viso gaudamas 593,76 ETH.

  \item \textbf{ETH konvertavimas į BUSD.}  
  (7)--(10) pavedimais likvidatorius keičia ETH į BUSD tam, kad galėtų grąžinti trumpalaikę paskolą pirmajai keityklai. Po dviejų keitimų gauta 1245496 BUSD.

  \item \textbf{Skolos grąžinimas keityklai.}  
  Kadangi BUSD-T buvo gautas pirmajame pavedime, (11) pavedimu 
  grąžina 1238753 BUSD (maždaug 99,45\% iš (10) gautos sumos) pirmajai keityklai.

  % \item \textbf{Pelnas likvidatoriui.}  
  % Po visų veiksmų likvidatoriaus piniginėje lieka 6743 BUSD, kurie (12-uoju pavedimu) pervedami į pelno piniginę. BUSD prilygsta JAV dolerį, todėl galutinės pajamos \$6743, kur mokestis už tranzakciją \$204, todėl pelnas \$6539.

  \item \textbf{Pelnas likvidatoriui.}  
Po visų veiksmų likvidatoriaus piniginėje lieka 6743 BUSD, kurie (12-uoju pavedimu) pervedami į pelno piniginę.  
Kadangi BUSD prilygsta JAV doleriui, galutinės pajamos yra 6743 JAV doleriai.  
Sumokėjus 204 JAV dolerių tranzakcijos mokestį, likvidatoriui lieka 6539 JAV dolerių pelnas.
\end{enumerate}

\noindent
Jeigu likvidatoriui nereikėtų mokėti valiutų konvertavimo mokesčių, 
jam užtektų grąžinti apie 90,91\% (likvidacija grąžina 110\% įdėtos vertės) 
gauto užstato po likvidacijos. Tačiau realybėje egzistuoja tiek 
konvertavimo mokesčiai, tiek rinkos kainos gali skirtis nuo orakulo 
duomenų, todėl likvidatoriui teko grąžinti apie 99,45\% galutinės 
gautos valiutos.

\subsubsection{Atkartojimas}

\begin{table}[h!]
  \centering
  \caption{Reikšmės atkartojus likvidaciją}
  \begin{tabular}{|l|l|}
    \hline
    Mokestis už vieną kuro vienetą                           & 0,000000316441261625 BNB        \\
  \hline
  \multicolumn{2}{|c|}{\textbf{Likvidacija}}                              \\ \hline
  Grąžinta suma                            & 1213110,4472758995334892 BUSD-T         \\ \hline
  Užstato gauta \textit{wrapped} formatu             & 29506,36409804 vETH                     \\ \hline
  Kuro sunaudota dėl grąžinamos sumos patvirtinimo            & 24263                              \\ \hline
  Kuro sunaudota likvidacijos iškvietimui              & 816562                             \\ \hline
  Grąžinamos valiutos kaina               & 1,00148308 \$/BUSD-T               \\ \hline
  Užstato valiutos kaina          & 2250,50651698 \$/ETH            \\ \hline
  \multicolumn{2}{|c|}{\textbf{Užstato išsigryninimas (\textit{redeem})}}                                         \\ \hline
  \textit{Gas} sunaudota išsigryninimui                   & 141054                             \\ \hline
  Užstato gauta           & 593,762633190749838213 ETH             \\ \hline
  BNB kaina                & 298,22744081 \$/BNB             \\ \hline
  Pajamos                & \$121357,088416942287              \\ \hline
  Pelnas                 & \$121264,427054685937748              \\ \hline
  \end{tabular}
  \label{liquidation_example_repeat}
  \end{table}

\ref{liquidation_example_repeat} lentelėje pateikti skaičiai gauti pakartojus likvidaciją. Matome, kad pajamos yra 121 tūkstantis, kur \ref{img:liquidation_example} pav. pajamos yra tiktais 6,7 tūkstančiai. Šis skirtumas atsiranda iš to, kad orakulo kaina nesutampa su tuo ką gavo likvidatorius atliekant konvertavimus. Paimkime net pirmą konvertavimą po likvidacijos - ETH į BNB. Pagal orakulo kainas likvidatorius turėjo gauti 4480,7 BNB, tačiau gavo 4176,95 (6,78\% mažiau). Tai gali būti tiek dėl orakulo ETH pervertinimo, o gal ETH-BNB keitykla tuo metu turėjo prastą kainą ir/arba mažą likvidumą. Pažymime, kad mūsų tyrimui šis skirtumas nėra svarbus, nes koncentruojamės tik į likvidacijos dalies optimizaciją ir proporciškai lyginame gautą užstatą.

\subsubsection{Strategijų lyginimas}
Toliau lyginamos skirtingos strategijos. Siekiant aiškiau suprasti situaciją, \ref{tab:strategijos_cf_analize} lentelėje pateikiama skolininko pradinė pozicija – nurodomos paskolos ir užstato valiutos, jų vertės bei atitinkami užstato koeficientai (CF). Remiantis \ref{liquidation_example_comp} lentelės duomenimis, matyti, kad originali likvidacija (\$1{,}21 mln.) buvo gerokai mažesnė nei maksimali galima suma (\$75{,}49 mln.) taikant strategiją \textbf{iki uždarymo ribos}. Strategija \textbf{didžiausia skola} sutampa su \textbf{iki uždarymo ribos}, nes skolininko pozicijoje buvo tik viena pasiskolinta valiuta, o istorinis likvidatorius pasirinko likviduoti didžiausią užstatą – ETH. Naudojant orakulo teikiamas valiutų vertes, buvo galima pasiekti iki \$7{,}54 mln. pelną, tačiau tai reikalautų didelės apimties valiutų konvertavimo. Tokios operacijos reikalauja labai likvidžių rinkų, o siekiant išvengti reikšmingų kainų svyravimų, konvertavimą galimai tektų skaidyti per kelias rinkas. Pastebėtina, kad net ir originalus likvidatorius, konvertavęs tik \$1{,}21 mln. vertės turtą, jau susidūrė su nepalankiomis kainomis. Likusių rinkų likvidumo įvertinti negalime, nes tai reikalautų papildomos istorinės tų metų rinkų analizės.

Vis dėlto buvo galima likviduoti dar didesnį kiekį nei \$75,49 mln., pasinaudojus pasiūlyta \textbf{pilno išeikvojimo} strategija. \ref{liquidation_example_comp} lentelėje šiai strategijai pateiktos dvi dalinės likvidacijos, išskirtos skolos grąžinimo ir užstato paėmimo stulpeliuose. Pirmosios likvidacijos dydis siekė \$36,27 mln., po jos skolininko sveikumo koeficientas turėjo likti šiek tiek mažesnis nei 1. Antroji likvidacija sudarė \$46,29 mln., tad bendra grąžinta skola iš viso siekė \$82,56 mln., o atitinkamai ir likvidatoriaus atlygis buvo proporcingai didesnis.

Iki šiol buvo aptariamos strategijos, kuriose tiek paskolos, tiek užstato valiutos buvo fiksuotos – pavyzdžiui, kai buvo likviduojama konkreti pora (\textit{USDT, ETH}). Taikant \textbf{nuo didžiausio užstato koeficiento} strategiją, paskolos valiuta (USDT) išlieka fiksuota, tačiau atsiranda galimybė pasirinkti, kurią užstato valiutą likviduoti. 

Taikant \textbf{nuo didžiausio užstato koeficiento} strategiją, buvo įvykdyti trys likvidacijos veiksmai, kiekvienas naudojant skirtingą užstato valiutą. Bendra sugrąžintos paskolos vertė siekė \$93{,}62 mln. – tai 13\,\% daugiau nei taikant \textbf{pilno išeikvojimo} strategiją. Atidžiau išanalizavus matyti, kad \textbf{pilno išeikvojimo} strategijoje naudotas užstatas visiškai sutampa su skolininko turimu ETH kiekiu. Tai reiškia, kad antroji likvidacija buvo apribota užstato kiekiu ir net nepasiekė uždarymo ribos. Vadinasi, apsiribojus vienos valiutos užstato likvidavimu, prarandama dalis galimo pelno. Priešingai, \textbf{nuo didžiausio užstato koeficiento} strategija leidžia naudoti kelias užstato valiutas, todėl nėra apribojama konkrečios valiutos kiekiu, o likvidacija vyksta tol, kol pasiekiama uždarymo riba. Tokiu būdu ši strategija leidžia efektyviau išnaudoti skolininko užstatus ir generuoja didesnę grąžą.

Apie \textbf{nuo didžiausio užstato koeficiento} atvejį dar galima pasakyti, kad XVS užstatas buvo nelikviduotas, nes jo vertė siekė tik apie 5 centus, o likvidacijos kvietimas būtų kainavęs daugiau nei gautas pelnas. Taip pat šiuo atveju visos skolininko pozicijoje esančios užstato valiutos turėjo vienodą užstato koeficientą (CF), todėl rikiavimas pagal koeficientą nedarė įtakos. Dėl to \textbf{nuo mažiausio užstato koeficiento} strategija generavo identišką pelną kaip ir \textbf{nuo didžiausio užstato koeficiento}.

\begin{table}[H]
\centering
\caption{Skolininko pradinė pozicija – skolos ir užstato duomenys}
\label{tab:strategijos_cf_analize}
\begin{tabular}{|c|c|c|c|}
\hline
\textbf{Valiuta} & \textbf{Skola (\$)} & \textbf{Užstatas (\$)} & \textbf{CF} \\ \hline
USDT &  150,97M   &  0        & 80\%  \\ \hline
XVS  &  0         &  0,05     & 80\%  \\ \hline
BTC  &  0         &  8,40M    & 80\%  \\ \hline
BNB  &  0         &  84,05M   & 80\%  \\ \hline
ETH  &  0         &  90,82M   & 80\%  \\ \hline
\end{tabular}
\end{table}

\begin{table}[h!]
  \centering
  \caption{Skirtingų likvidavimo strategijų rezultatai}
  \begin{tabular}{|>{\raggedright\arraybackslash}m{3.5cm}|>{\centering\arraybackslash}p{2cm}|>{\centering\arraybackslash}p{4cm}|c|c|}
  \hline
  \textbf{Strategija} & \textbf{Grąžinimas} & \textbf{Paimtas užstatas} & \textbf{Mokestis už kurą} & \textbf{Pelnas} \\ \hline
  Atkartoti                                    & \$1,21M   & 594 ETH (\$1,34M)     & 981 879 (\$92,66)    & \$121,26K \\ \hline
  Iki uždarymo ribos / Didžiausia skola        & \$75,49M & 36,9K ETH (\$83,03M) & 981 888 (\$92,66)    & \$7,54M   \\ \hline
  Pilnas išeikvojimas                & \makecell[c]{\$36,27M \\ \$46,29M \\ = \$82,56M}  & \makecell[c]{17,7K ETH (\$39,89M) \\ 22,6K ETH (\$50,92M) \\ = 40,3K ETH (\$90,81M)}  & 1 630 707 (\$153,89) & \$8,25M   \\ \hline
  Nuo didžiausio/mažiausio užstato koeficiento & \makecell[c]{\$7,6M \\ \$28,6M \\ \$57,3M \\ = \$93,62M}  &  \makecell[c]{231 BTC (\$8,40M) \\ 105K BNB (\$31,49M) \\ 28K ETH (\$63,07M) \\ = \$102,97M}    & 2 665 850 (\$251.58)   & \$9,35M   \\ \hline
  \end{tabular}
  \label{liquidation_example_comp}
  \end{table}

\subsection{Pilnas išeikvojimas vienodoms valiutoms}
Parodome, kad pilno išeikvojimo strategija gali būti efektyvesnė nei primityvioji \textit{didžiausios skolos} strategija, kartu eliminuodami valiutos likvidumo klausimą. Šiame skyriuje nagrinėjame tik tuos atvejus, kai grąžinimo ir užstato valiutos sutampa, todėl ir pelnas bus išreikštas ta pačia valiuta.

Arbitražo vykdymo kontekste likviduojama valiuta nebūtinai yra ta, kurią likviduotojas ketina laikyti ilgesnį laiką. Todėl, įvykdęs likvidaciją, likviduotojas gali gautą pelną ta valiuta konvertuoti į stabilesnę, pavyzdžiui, USDC, kuris yra susietas su JAV doleriu. Šis konvertavimą galima atlikti atskiroje tranzakcijoje, kurios kuro mokestis už vienetą yra mažesnis nei pačios likvidacijos metu.

Didelis kuro mokestis likvidacijos tranzakcijai gali būti nustaytas dėl kelių priežasčių. Viena priežastis – noras būti greičiau įtrauktam į bloką arba būti pirma tranzakcija bloke. Kita priežastis – pasinaudoti blokų grandinės tranzakcijų rikiavimo logika: nustatant aukštesnį kuro mokestį galima valdyti tranzakcijos vietą bloke, siekiant, kad ji būtų iškart po kitos tranzakcijos, sukuriančios likvidacijos galimybę, pavyzdžiui, orakulo kainos atnaujinimo.

Kalbant apie skolos grąžinimą, būtina turėti kapitalo, reikalingo skolai padengti. Šį kapitalo prieinamumo klausimą sprendžia įvairios rinkos, leidžiančios momentiškai (per \textit{flash loan}) pasiskolinti konkrečią valiutą už tam tikrą mokestį. Taigi, pasitelkus šias priemones ir technikas, galima vykdyti likvidacijas su minimalia rizika bei gauti pelną.

\subsubsection{Pelningiausias atvejis}

% SELECT (drain_same_token->>'profit_usd')::NUMERIC, transaction_hash, drain_same_token, created_at
% FROM bsc.venus_liquidation_tests
% WHERE TRUE
% AND (drain_same_token->>'profit_usd')::NUMERIC > 0
% ORDER BY 1 DESC

Iš išanalizuotų likvidacijų identifikuotas pelningiausias \textbf{pilno išeikvojimo vienodoms valiutoms} strategijos atvejis. Analizuojame likvidaciją \\ \texttt{0xcebfe3ec5782ad5f1a52f3727c3986abb1f2314cd1c0f0baf18ec20e1000fea9}.

\ref{tab:drain_same_token_position_most} lentelėje pateikta skolininko pozicija – ji sudaryta tik iš DOGE valiutos. Dėl šios priežasties daugelio strategijų veiksmai sutampa. \ref{tab:drain_same_token_profit_most} lentelėje pateikti skirtingų strategijų pelnai.

Šiuo atveju originalus likvidatorius likvidavo iki uždarymo ribos, todėl jo pelnas sutampa su \textbf{iki uždarymo ribos} strategijos rezultatu. \textbf{Didžiausia skola} strategija taip pat duoda tokį patį rezultatą, kadangi nėra kitų galimų valiutų porų (tik DOGE-DOGE).

\textbf{Pilno išeikvojimo} strategijai pavyko likviduoti didesnį kiekį skolos nei paprastai \textbf{iki uždarymo ribos} strategijai, todėl ji uždirbo apie 20\% didesnį pelną. Kaip matyti ir iš lentelės, skolininko sveikatos faktorius ($HF$) po likvidacijos buvo gerokai aukštesnis, todėl ši strategija pasirodė efektyvesnė likviduojant kuo didesnę skolos dalį.

\begin{table}[H]
\centering
\caption{Pradinė skolininko pozicija}
\label{tab:drain_same_token_position_most}
\begin{tabular}{|c|c|c|c|c|}
\hline
\textbf{Valiuta} & \textbf{Skola (\$)} & \textbf{Užstatas (\$)} & \textbf{CF} & \textbf{HF} \\ \hline
DOGE &  2 617 497   &  5 822 673       & 40\% & 0,889 \\ \hline
\end{tabular}
\end{table}

\begin{table}[h!]
  \centering
  \caption{Skirtingų likvidavimo strategijų rezultatai}
  \begin{tabular}{|>{\raggedright\arraybackslash}m{3.5cm}|>{\centering\arraybackslash}p{2.9cm}|>{\centering\arraybackslash}p{2.5cm}|>{\centering\arraybackslash}p{2.5cm}|c|>{\centering\arraybackslash}p{2cm}|}
  \hline
  \textbf{Strategija} & \textbf{Grąžinimas} & \textbf{Paimtas užstatas} & \textbf{Mokestis už kurą} & \textbf{Pelnas} & \textbf{Galutinis $HF$} \\ \hline
  Atkartoti & \$1,308M   & \$1,439M     & \$104    & \$130K & 1.339 \\ \hline
  Iki uždarymo ribos / Didžiausia skola & \$1,308M   & \$1,439M     & \$104    & \$130K & 1.339 \\ \hline
  Pilnas išeikvojimas / Pilnas išeikvojimas vienodoms valiutoms & \makecell[c]{\$515K \\ \$1,051M \\ = \$1,566M}   & \makecell[c]{\$566K \\ \$1,156M \\ = \$1,722M}     & \$171    & \$156K & 1.56  \\ \hline
  \end{tabular}
  \label{tab:drain_same_token_profit_most}
  \end{table}

\subsubsection{Proporcingai pelningiausias su didziausia skola}
% SELECT
% (large_borrow ->>'profit_usd')::NUMERIC AS largest_loan_profit,
% (drain_same_token->>'profit_usd')::NUMERIC AS same_token_profit,
% transaction_hash, drain_same_token, created_at
% FROM bsc.venus_liquidation_tests
% WHERE TRUE
% AND (drain_same_token->>'profit_usd')::NUMERIC > 10000
% AND (large_borrow ->>'profit_usd')::NUMERIC  > 0
% ORDER BY (drain_same_token->>'profit_usd')::NUMERIC / (large_borrow ->>'profit_usd')::NUMERIC  DESC

\subsubsection{Bendras pelnas}
% SELECT count(*) AS total_liquidations,
% count(*) FILTER (WHERE (drain_same_token->>'profit_usd')::NUMERIC > 0) AS drain_same_token_profitable,
% count(*) FILTER (WHERE (drain_same_token->>'profit_usd')::NUMERIC > (large_borrow->>'profit_usd')::NUMERIC) AS drain_same_token_more_profitable_than_largest_borrow
% FROM bsc.venus_liquidation_tests

Iš viso išanalizuota 23\,202 likvidacijų. Iš jų 2\,382 atvejais strategija \textbf{pilnas išeikvojimas vienodoms valiutoms} buvo pelninga. Iš šių atvejų 311 parodė didesnį pelną nei strategija \textbf{didžiausia skola}.

Vertinant bendrą pelną pasirinktai strategijai, naivu būtų kiekvieną įvykusią likvidaciją analizuoti atskirai, apskaičiuoti pelną pagal pasirinktą strategiją ir šiuos rezultatus tiesiog sumuoti. Kaip matyti iš \ref{liquidation_example_comp} lentelės, istoriniuose duomenyse ne visi likvidatoriai maksimaliai išnaudojo turimą likvidacijos potencialą, o didesnės galimybės dažnai būna išskaidytos į daugybę mažesnių likvidacijų. Dėl to būtų klaidinga sumuoti pelnus kiekvienai atskirai likvidacijai, nes geresnės strategijos atveju tas pats pelnas gali būti įtrauktas kelis kartus. Problema kyla dėl to, kad viena likvidacija gali turėti įtakos kitoms, su ja susijusioms likvidacijoms – modifikuodami vieną iš jų, galime netiesiogiai pakenkti kitoms. Savo tyrime kiekvieną atvejį analizuojame izoliuotai, neatsižvelgdami į jų tarpusavio sąveikas. Dažniausias ryšys tarp skirtingų likvidacijų yra tas pats skolininkas. Kiti galimi priklausomumai – likvidacija, kuri išnaudoja visą likusį grynųjų likutį iš \textit{Venus} valiutų baseino, atima galimybę išsigryninti kitiems dalyviams; arba vieno likvidatoriaus veiksmai sumažina rinkos likvidumą, apsunkindami valiutų keitimą kitiems.

Norėdami įvertinti strategijos \textbf{pilnas išeikvojimas vienodoms valiutoms} bendrą pelningumą, tais atvejais, kai tam pačiam skolininkui galimi keli įvykiai, taikome apribojimą: kiekvienam skolininkui į pelno skaičiavimą įtraukiamas tik vienas – pelningiausias – įvykis. Pritaikius šį apribojimą, pelningų atvejų skaičius sudaro $1304$ iš $2382$.

% WITH most_profitable_liquidation_tests AS (
%     SELECT DISTINCT ON (borrower) *
%     FROM bsc.venus_liquidation_tests
%     JOIN bsc.venus_liquidations USING(transaction_hash)
%     WHERE (drain_same_token ->> 'profit_usd')::NUMERIC > 0
%     ORDER BY borrower, (drain_same_token ->> 'profit_usd')::NUMERIC DESC
% ),
% assets AS (
%     SELECT 
%       transaction_hash, jsonb_array_elements(drain_same_token->'assets') AS asset
%     FROM most_profitable_liquidation_tests 
%     WHERE TRUE
%     AND (drain_same_token ->>'profit_usd')::NUMERIC  > 0
% ),
% aggregated_assets AS (
%     SELECT
%     asset->'initial_data'->>'symbol' AS symbol,
%     sum((asset->>'collateral_underlying_gained')::NUMERIC) AS collateral_gained,
%     sum((asset->>'repaid')::NUMERIC) AS repaid,
%     sum((asset->>'collateral_underlying_gained')::NUMERIC - (asset->>'repaid')::NUMERIC) AS profit_underlying,
%     sum(((asset->>'collateral_underlying_gained')::NUMERIC - (asset->>'repaid')::NUMERIC) * (asset->'initial_data'->>'price')::NUMERIC / 1e36) AS profit_usd,
%     sum((asset->>'liquidations_participated')::NUMERIC) AS liquidations_participated
%     FROM assets a
%     WHERE (asset->>'collateral_underlying_gained')::NUMERIC > 0
%     AND (asset->>'repaid')::NUMERIC > 0
%     GROUP BY asset->'initial_data'->>'symbol'
% )
% --,res AS (
% SELECT
%     symbol,
% --    collateral_gained,
% --    repaid,
% --    profit_underlying,
%     profit_usd,
%     liquidations_participated
% FROM aggregated_assets a
% ORDER BY profit_usd DESC

% )
% SELECT sum(profit_usd) FROM res

\ref{tab:liq_same_tokens_profit} lentelėje pateikiamas \textbf{pilnas išeikvojimas vienodoms valiutoms} – tai bendras pelnas, gautas per kiekvieną valiutą. Daroma prielaida, kad pelnyta valiuta buvo parduota pagal to meto orakulo kainą, o mokesčiai yra nereikšmingi. Bendras pelnas siekia 1,4 mln. JAV dolerių, t. y. teoriškai buvo galima uždirbti bent tiek, vykdant paprastas likvidacijas, kai grąžinimo ir atsiimamo užstato valiutos yra vienodos. Tokiu atveju galima išvengti galimai nuostolingų valiutų konvertavimų likvidacijos metu bei su tuo susijusių mokesčių – tiek valiutų keitimo, tiek sunaudotų kuro vienetų, ypač esant aukštoms kuro kainoms.

\begin{table}[H]
\centering
\caption{\textit{Pilnas išeikvojimas vienodoms valiutoms} pelnas pagal valiutą}
\label{tab:liq_same_tokens_profit}
\begin{tabular}{|l|r|r|}
\hline
\textbf{Valiuta} & \textbf{Valiutos vertė (\$)} & \textbf{Likvidacijų kvietimai} \\
\hline
DOGE  & 333 188.40  & 21 \\
BUSD  & 280 728.73  & 301 \\
SXP   & 169 691.27  & 57 \\
ETH   & 134 977.21  & 120 \\
DAI   & 119 947.04  & 26 \\
BTC   & 91 450.61   & 173 \\
USDT  & 85 647.06   & 321 \\
BNB   & 81 680.97   & 374 \\
USDC  & 80 301.13   & 174 \\
XVS   & 20 878.56   & 14 \\
MATIC & 12 799.84   & 10 \\
AAVE  & 8 226.96    & 7 \\
DOT   & 1 654.70    & 19 \\
XRP   & 1 519.91    & 23 \\
ADA   & 1 145.77    & 17 \\
BETH  & 899.91     & 30 \\
CAKE  & 895.97     & 11 \\
BCH   & 614.01     & 5 \\
WBETH & 484.21     & 2 \\
CAN   & 349.34     & 9 \\
FIL   & 135.10     & 5 \\
LTC   & 89.87      & 11 \\
LINK  & 60.36      & 8 \\
TRX   & 50.41      & 1 \\
TRXOLD & 43.02     & 8 \\
UST   & 24.71      & 1 \\
TUSD  & 3.79       & 1 \\
\hline
\end{tabular}
\end{table}

\subsection{Nuo didžiausio užstato koeficiento}
\textcolor{red}{
Palyginti "nuo didžiausio" ir "nuo mažiausio". Turėtų būti skirtumas kai skolininko pozicijos turi valiutų su skirtingais koeficientais.
}

\subsection{Strategijų vertinimas remiantis dideliais duomenų rinkiniais}
\label{sec:lyginimas_daug}

\textcolor{red}{
Papildyti su naujomis strategijomis: didziausia skola, nuo didziausio/maziausio cf, pilno iseikvojimo su vienodomis valiutomis.
}

% Skirtingoms strategijoms lyginti patogu naudoti kaupiamojo pelno kreivę. Galime analizuoti kiekvieną įvykusią likvidaciją, apskaičiuoti pelną pagal skirtingas strategijas ir laikui bėgant sumuoti pelnus. Tačiau, kaip matome iš \ref{liquidation_example_comp} lentelės, istorijoje ne visi likvidatoriai maksimaliai išnaudojo leidžiamą likvidacijų potencialą, o didelės galimybės dažnai yra suskaidomos į daug mažesnių likvidacijų. Dėl šios priežasties būtų klaidinga sumuoti pelnus kiekvienai įvykusiai likvidacijai, nes pelnai pagal geresnes strategijas būtų skaičiuojami kelis kartus. Problema kyla iš to, kad likvidacija turi įtakos kitoms su ja susijusioms likvidacijoms ir joms gali pakenkti jeigu pamodifikuotume ankstesnę likvidaciją. Dažniausias ryšys tarp skirtingų likvidacijų – tas pats skolininkas. Kiti galimi susiejimai: viena likvidacija išnaudoja visą likusį grynųjų likutį iš \textit{Venus} valiutos baseino, todėl kiti likvidatoriai netenka galimybės gauti grynųjų; likvidatoriai išnaudoja rinkos likvidumą keisdami valiutas.

Norėdami ir tokioje situacijoje pavaizduoti kaupiamąją pelno kreivę, laikysimės kelių apribojimų: analizuosime tik vieną kiekvieno skolininko likvidaciją istorijoje, pasirinkdami pirmąją, kurios metu buvo galima visiškai išgryninti visą gautą užstatą visomis trimis strategijomis. Laikantis šių apribojimų gauname 10892 įvykių iš 63982 analizuojamų (17,02\%). Taigi atlikę kiekvienai istorinei likvidacijai analizę kaip \ref{liquidation_example_comp} lentelėje ir susumavę pelną per laiką (kas atitinka ėjimą per blokus) gauname rezultatą \ref{img:bendras2} pav.

\begin{figure}[H]
  \centering
  \includegraphics[scale=0.4]{img/bendras4.png}
  \caption{Kaupiamasis pelnas pagal strategijas, atsižvelgiant tik į pirmąją kiekvieno skolininko likvidaciją}
  \label{img:bendras2}
\end{figure}

\begin{table}[h!]
  \centering
  \caption{Strategijų pelnai}
  \label{tab:strategiju_pelnai}
  \begin{tabular}{|l|r|}
  \hline
  \textbf{Strategija}                     & \textbf{Pelnas (\$)} \\ \hline
  Atkartoti                               & 20 355 860,62        \\ \hline
  Iki uždarymo ribos              & 32 149 135,99        \\ \hline
  Pilnas išeikvojimas                           & 35 089 532,28        \\ \hline
  \end{tabular}
  \end{table}

Rezultatai \ref{tab:strategiju_pelnai} lentelėje rodo, kad laikantis nurodytų ribojimų \textit{iki uždarymo ribos} strategija generavo 58\% didesnį pelną nei \textit{atkartoti}, o \textit{pilnas išeikvojimas} strategija – net 72\% daugiau. Pastebėtina, kad reikšminga dalis prarasto pelno yra susijusi su tuo, jog likvidatoriai dažnai negrąžina maksimalios leistinos sumos. Be to, yra maždaug 9\% potencialas padidinti pelną naudojant efektyvesnį likvidacijos algoritmą.

Galime atkreipti dėmesį, kad \ref{img:bendras2} pav. visų strategijų pelno kreivės yra panašios, jei atmesime kelis išskirtinius staigius šuolius tam tikruose laiko momentuose. Tai reiškia, kad strategijų pelno skirtumą smarkiai veikia keletas įvykių.

Nuo 18259454 \cite{LikvidacijosKontraktas} bloko \textit{Venus} protokolas pakeitė likvidavimo mechanizmą. Po pakeitimo visos likvidacijos privalo būti vykdomos per \textit{Venus} protokolo parengtą kontraktą, kuris pasilieka pusę likvidacijos pelno, 5\% nuo likviduojamos sumos, ir kitus 5\% atiduoda likvidatoriui. Po šio pakeitimo pastebimas reikšmingas pelno lėtėjimas, tačiau tai taip pat gali būti susiję su tuo, kad laikui bėgant į grafiką patenka vis mažiau likvidacijų, nes analizuojame tik po vieną likvidaciją vienam skolininkui.

\subsubsection{Išskirtiniai atvejai}
\textcolor{red}{
Papildyti su "nuo didžiausio/mažiausio užstato koeficiento" pelnais.
}

Blokų ruože 7544850–7546511 įvyko trys didelės likvidacijos, kurių metu pastebėtas reikšmingas skirtumas tarp strategijų:

\begin{enumerate}[label=\textbf{\Alph*.}]
    \item 0x0cd0fc0cdd5b572d71cd039cc522d20dfcc5c8c0772173b484c91194401fe89b – blokas 7544850
    \item 0x718cf2813f3124f576a64a69429ec543ea6b14ca53d557772d61e72a6c256f3e – blokas 7546281
    \item 0x3b04a03ed356108c7297e6b438d70df7383f10d39d0511603b576b635d6bff9f – blokas 7546511
\end{enumerate}

\begin{table}[h!]
  \centering
  \caption{Strategijų pelno palyginimas (skliaustuose likvidacijų iškvietimų skaičius)}
  \label{tab:profit_table}
  \begin{tabular}{|l|c|c|c|}
  \hline
  \textbf{Strategija}        & \textbf{A}          & \textbf{B}          & \textbf{C}          \\ \hline
  Atkartoti                  & \$2 015 683      & \$190 227       & \$15 074        \\ \hline
  Iki uždarymo ribos         & \$2 015 683      & \$6 630 951      & \$1 060 317      \\ \hline
  Pilnas išeikvojimas        & \$4 031 270 (50)      & \$6 630 951 (1)      & \$1 083 910 (2)      \\ \hline
  \end{tabular}
  \end{table}

Įdomu yra tai, kad visi trys nagrinėjami atvejai išsiskiria savo duomenimis dėl skirtingų priežasčių:

\begin{enumerate}[label=\textbf{\Alph*.}]
  \item Pirmuoju atveju originalioje tranzakcijoje likvidatorius grąžino maksimalią leistiną sumą per vieną likvidacijos funkcijos iškvietimą. Dėl šios priežasties \textit{atkartoti} ir \textit{iki uždarymo ribos} strategijų pelnai buvo identiški. Šiame scenarijuje skolininkas buvo reikšmingai įsiskolinęs, todėl likvidavimo procesas galėjo būti tęsiamas net 49 kartus, kiekvieną kartą grąžinant dvigubai mažesnę sumą nei ankstesnėje iteracijoje. Galutinis pelnas, pasiektas vykdant šį procesą, buvo beveik du kartus didesnis už \textit{iki uždarymo ribos} strategijos pelną.

  \item Antruoju atveju likvidatorius pasirinko likviduoti žymiai mažesnę sumą, nei leido protokolo uždarymo riba. Dėl to \textit{iki uždarymo ribos} strategijos pelnas buvo maždaug 35 kartus didesnis nei \textit{atkartoti} strategijos. Šiame kontekste likvidatorių ribojo ne uždarymo riba, o skolininko turimo užstato kiekis. Todėl \textit{Pilnas išeikvojimas} strategija šioje situacijoje reikalavo tik vieno likvidacijos iškvietimo, kad būtų maksimaliai išnaudota galimybė.

  \item Trečiuoju atveju, kaip ir antrajame, originalus likvidatorius grąžino mažesnę sumą nei leido protokolo nustatyta uždarymo riba. Tačiau šiuo atveju likvidatorių ribojo būtent uždarymo riba. Siekiant optimizuoti pelną, buvo atliktos dvi atskiros likvidacijos, kurios leido papildomai sugeneruoti \$23593 virš \textit{iki uždarymo ribos} strategijos pelno.
\end{enumerate}

\subsubsection{Atmestos likvidacijos}

Atkartojimo strategija pasirodė esanti veiksminga, nes buvo pastebėta, kad ne visas likvidacijas pavyktų atkartoti bet kuriam likvidatoriui. Analizės metu buvo atmestos likvidacijos, kurios turėjo privilegijuotą statusą ir galimybę jas vykdyti turėjo tik tam tikri adresai, galėję iškviesti likvidavimo funkciją:

\begin{enumerate}
    \item Likvidacijos, kurios toje pačioje tranzakcijoje atliko platformos konfigūracijos pakeitimus, leidžiančius likviduoti tam tikrus skolininkus:
    \begin{itemize}
        \item \texttt{0x7c97317afe5911e704bd684e8b3fe472d7b8703b54321ab564be2bbeacdb0f5f} – VIP-36 Refactor SXP \& XVS Collateral Factor and Interest Rate Model Change
        \item \texttt{0xc81fa724698490d096b04cccb080195517f4df5cfa56121cbee895d05ad0de53} – VIP-37 Refactor SXP \& XVS Collateral Factor and Reward Speed
        \item \texttt{0xb18543cd79c90ef2ca1e463aaf3760e6e4e731b7fa64a86e6f2538de392d49df} – VIP-223 Risk Parameters Adjustments (BUSD)
    \end{itemize}
    \item \textit{BNB bridge exploiter} likvidacijos, kur tik platformos savininkai sau leido likviduoti:
    \begin{itemize}
        \item Skolininko adresas: \texttt{0x489a8756c18c0b8b24ec2a2b9ff3d4d447f79bec}
        \item Viena iš 14 likvidacijų - \\0xc4bd0beaedfa6985c7976d6c1dd681ec1e4fa805067572e95beae32869c88cd7
    \end{itemize}
\end{enumerate}

Taip pat, nebuvo analizuojamos likvidacijos, kurios išnaudojo \textit{Venus} protokolo mechanizmo klaidą, susijusią su paskatų kaupimu naudojantis protokolu. Naudotojai, skolindami arba skolindamiesi protokole, kaupė atlygį, kurį galėjo atsiimti iškviesdami funkciją \textbf{claimVenus}. Įdomu tai, kad šią funkciją galėjo iškviesti bet kuris adresas kito naudotojo vardu. Tai suteikė galimybę išnaudotojams aptikti adresus, kurie buvo stipriai įsiskolinę ir sukaupę nemažą atlygį, atsiimti sukauptą atlygį jų vardu ir tuoj pat inicijuoti jų likvidaciją. Vienas tokio išnaudojimo pavyzdys užfiksuotas tranzakcijoje: 0x801001726f7c0c2434a8ea1680213ebfd5201094087c94d7dac44b7860555f1c. Šis pažeidžiamumas vėliau buvo pašalintas, įdiegiant apribojimą, kad \textit{claimVenus} funkcija negali būti sėkmingai iškviečiama, jei pozicija yra likviduojama ($HF < 1$) \cite{exploitFix}.

% \section{Formatavimas}

% Šiame skyriuje bus pateikti pavyzdžiai matematinio teksto, lentelių ir paveikslėlių formatavimams bei aprašyta, kaip taisyklingai suformuluoti matematinius jūsų baigiamojo darbo rezultatus.

% \subsection{Matematinis tekstas}

% Matematinės formulės gali būti įterptos teksto pastraipose, formulės \LaTeX~kodą atskiriant simboliais \texttt{\$...\$}. Pavyzdys: trigonometrinė tapatybė $\sin^2 \alpha + \cos^2 \alpha = 1$.

% Tačiau formulės atrodys daug gražiau, jeigu jos bus išskirtos į atskiras lygtis, formulės kodą patalpinant į aplinką \texttt{\textbackslash[...\textbackslash]}. Štai tokios lygties pavyzdys:
% \[
% \ee^{i \alpha} = \cos{\alpha} + i \sin{\alpha}, \qquad \alpha \in \RR.
% \]
% Šioje lygtyje buvo panaudoti matematiniai simboliai $\RR$ ir $\ee$, kurių komandos \texttt{\textbackslash RR} ir \texttt{\textbackslash ee} apibrėžtos šablono pradžioje.

% Kartais formulės užima kelias eilutes, pvz.:
% \begin{equation}
% \begin{split}
% 2&= 1+1+0=\left(\frac{\sqrt{16}}{\tan^2\pi/3+1}\right) +\ln\ee+\sin\pi\\
% &= (\sin^2 17+\cos^2 17)^{\ln\ee}+\cos 0 +(x^{1/\ln x})'. 
% \label{form1}
% \end{split}
% \end{equation}

% Nepamirškite padėti taško ($.$) formulės pabaigoje, jeigu tai sakinio pabaiga. Taip pat akreipkite dėmesį į skliaustų, kurių viduje stovi didelė trupmena \texttt{\textbackslash frac}, aukštį, kuris automatiškai reguliuojamas komandomis \texttt{\textbackslash left( ... \textbackslash right)} arba nurodomas komandomis \texttt{\textbackslash big}, \texttt{\textbackslash Big}, \texttt{\textbackslash bbig}.

% \bigskip

% Jeigu formulės prisireiktų vėliau, jos nereikia kiekvieną kartą perrašinėti iš naujo. Reikiamą formulę visada galima pacituoti su komandą \texttt{\textbackslash eqref}. Pavyzdžiui, aukščiau užrašyta formulė su numeriu cituojama taip: lygtis \eqref{form1}. Tam reikia komanda \texttt{\textbackslash  label} formulei priskirti laikiną pavadinimą, kurį \LaTeX~automatiškai pakeis į reikiamą numerį. Daugiau informacijos apie \LaTeX~matematinius simbolius, lygtis, matematines aplinkas ir komandas galima rasti šiame dokumente \cite{amsdoc}.

% \bigskip

% Pateiksime dar keletą formulių, kuriose naudojamos sudėtingesnės matematinės komandos. Matricos ir determinantai užrašomi naudojant LaTeX aplinkas \texttt{pmatrix} ir \texttt{vmatrix}:
% \[
% A= \begin{pmatrix}
%     0 & 1\\
%     2 & 3
% \end{pmatrix}, \qquad
% \det A =
% \begin{vmatrix}
% 0 & 1\\
% 2 & 3    
% \end{vmatrix} = 0 \cdot 3 - 1 \cdot 2 = -2.
% \]
% Sudėtingesnėms lygtims ir matricoms formatuoti labai praverčia paketo \texttt{mathtools} \cite{mtoolsdoc} komandomis. Paketas \texttt{mathtools} įtrauktas į darbo šabloną, todėl jo komandomis galima naudotis tiesiogiai.

% Išvestinė užrašoma naudojant apostrofo simbolį (\texttt{'}), pavyzdžiui,
% \[
% (f(x)g(x))' = f'(x)g(x) + f(x)g'(x).
% \]
% Teiloro polinomas:
% \[
% p(x) = p(a) + p'(a)(x-a)+\frac{p''(a)}{2!}(x-a)^2 + ... + \frac{p^{(n)}}{n!}(x-a)^n.
% \]
% Paprastoms ir dalinėms išvestinėms, diferencialams, gradientams ir pan. užrašyti į darbo šabloną įtrauktos labai patogios komandos \texttt{\textbackslash dv} ir \texttt{\textbackslash pdv}, \texttt{\textbackslash dd}, \texttt{\textbackslash grad} iš \texttt{physics} paketo \cite{physdoc}:
% \[
% \dv{f}{x},  \qquad
% \dv[2]{f}{x}, \qquad
% \pdv{f}{x},  \qquad
% \pdv[5]{f}{x} \qquad
% \pdv{f}{x}{y}, \qquad
% \dd{f}, \qquad
% \grad{f}
% \]

% Integralą su rėžiais užrašysime naudodami \LaTeX komandą \texttt{\textbackslash int\textunderscore \{a\}\^{}\{b\}}:
% \[
% \int_{a}^{b}f(x) \dd x = F(a) - F(b) = \eval{F(x)}_{a}^{b}
% \]
% Daugialypiams, paviršiniams, kreiviniams integralams užrašyti galima naudoti komandas \texttt{\textbackslash iint}, \texttt{\textbackslash iiint}, \texttt{\textbackslash oint}, ir pan.
% \[
% \iint_{D}f(x, y)\dd{x}\dd{y},\quad
% \iint_{D} f(x,y)\dd{S}, \quad
% \int_{\gamma} f(x,y)\dd{l},\quad
% \oint_{\gamma} P(x,y)\dd{x}+Q(x,y)\dd{y}.
% \]

% \subsection{Matematinių rezultatų formulavimas}

% Jūsų darbo matematiniams rezultatams suformuluoti reikėtų naudoti aplinkas
% \[
% \text{\emph{Apibrėžimas}},\qquad \text{\emph{Teiginys}}, \qquad \text{\emph{Teorema}}, \qquad \text{\emph{Lema}},
% \]
% \[
% \text{\emph{Išvada}}, \qquad \text{\emph{Pastaba}}, \qquad  \text{\emph{Pavyzdys}}, \qquad \text{\emph{Įrodymas}}.
% \]
% Šios aplinkos jau yra apibrėžtos
% jūsų baigiamojo darbo šablone \texttt{VUMIFTemplateClass.cls}, sulietuvinus standartines \LaTeX\  komandas
% \[
% \texttt{definition}, \qquad \texttt{proposition}, \qquad \texttt{theorem}, \qquad \texttt{lemma},
% \]
% \[
% \texttt{corollary}, \qquad \texttt{remark}, \qquad \texttt{example}, \qquad \texttt{proof}.
% \]
% \noindent Apibrėžimo pavyzdys:
% \begin{definition}
% Skaičius $p \in \mathbb{N}$ yra vadinamas \emph{pirminiu skaičiumi}, jeigu jisai dalijasi tik iš $1$ ir savęs paties. Pirminių skaičių aibė yra žymima $\mathbb{P}$.
% \end{definition}
% \noindent Teiginio pavyzdys:
% \begin{proposition}
% Dviejų nepriklausomų atsitiktinių dydžių $X, Y: \Omega \to \mathbb{R}$ sandaugos $XY$ vidurkis lygus tų pradinių dydžių vidurkių sandaugai:
% \[
% \EE{(XY)}=\int_{\Omega} X(\omega)Y(\omega)\dd\mu(\omega) = \EE{X} \cdot \EE{Y},
% \]
% su sąlyga, kad $X$, $Y$ ir $XY$ vidurkiai egzistuoja.
% \end{proposition}

% \noindent  Svarbūs matematiniai teiginiai yra vadinami \emph{teoremomis}:
% \begin{theorem}[Pirmoji teorema apie izomorfizmą]\label{teor1}
%     Sakykime, kad $f: G {\rightarrow} H$ – grupių G ir H homomorfizmas. Tada grupės $G$ vaizdas $f(G)$ izomorfiškas faktorgrupei $G / \ker{(f)}$, tai yra
%     \[
%         f(G) \cong G \big / \ker{(f)}.
%     \]
% \end{theorem}

% \noindent Trumpesni pagalbiniai teiginiai vadinami \emph{lemomis}. Tačiau ir lemų formuluotės gali būti pakankamai sudėtingos:
% \begin{lemma}[Lema apie vektorių pakeitimą]\label{lem1}
%     Tarkime, kad tiesinės erdvės V virš kūno k vektoriai
%     \begin{equation}\label{šeima1}
%         v_1, v_2, \dots, v_s
%     \end{equation}
%     yrs tiesiškai nepriklausomi, ir kad kiekvienas šios šeimos vektorius $v_i$, $1 \leq i \leq s$ tiesiškai išreiškiamas vektoriais
%     \begin{equation}\label{šeima2}
%         w_1, w_2, \dots, w_t.
%     \end{equation}
%     Tuomet $s \leq t$, ir egzistuoja toks vektorių šeimos \eqref{šeima2} pošeimis $w_{j_1}, w_{j_2} ,  . . . , w_{j_s}$, kurį pakeitę vektoriais $v_1, v_2, . . . , v_s$, gausime vektorių šeimai \eqref{šeima2} ekvivalenčią šeimą \eqref{šeima2}.
% \end{lemma}

% \noindent Aplinka \emph{Pastaba} skirta smulkiems pastebėjimams:

% \begin{remark}
% Teoremos sąlyga, kad intervalas $[a, b]$ būtų kompaktiškas, o funkcija $f(x)$ tame intervale būtų tolydi, yra būtina.
% \end{remark}

% \noindent Kita aplinka \emph{Pavyzdys}, skirta trumpiems skaitiniams arba formulių pavyzdžiams:

% \begin{example}
% Lygčių sistemos
% \[
% \left\{
% \begin{array}{rclcl}
%     ax & + & by & = & e\\
%     cx & + & dy & = & f
% \end{array}
% \right.
% \]
% sprendinių Kramerio formulė:
% \[
% x = \frac{D_{x}}{D}, \qquad y = \frac{D_{y}}{D},\]
% čia
% \[
% D=
% \begin{vmatrix}
% a & b\\
% c & d    
% \end{vmatrix}=ad-bc, \qquad
% D_x=
% \begin{vmatrix}
% e & b\\
% f & d    
% \end{vmatrix}=ed-bf, \qquad
% D_y=
% \begin{vmatrix}
% a & e\\
% c & f    
% \end{vmatrix}=af-ec.
% \]
% \end{example}

% % \noindent Įrodymams užrašyti naudojama sulietuvinta aplinka \texttt{proof}. Žemiau užrašyti teiginys ir to teiginio įrodymas. Įrodymo pabaigą \LaTeX\ automatiškai pažymi \square\ simboliu.
% % \begin{proposition}
% %     Kvadratinė matrica $A$ yra neišsigimusi tada ir tik tada, kai $\det A \ne 0$.
% % \end{proposition}
% % \begin{proof}
% %     Jeigu $A$ yra neišsigimusi, tai egzistuoja matrica $B$, tokia kad $AB=I$. Remiantis matricų sandaugos determinanto savybe,
% %     \[
% %         \det A \cdot \det B = \det AB = \det I = 1.
% %     \]
% %     Taigi, $\det A \ne 0$.
% %     Dabar tarkime, kad $\det A \ne 0$. Tegul $A^*$ yra transponuota adjunktų matrica. Tuomet:
% %     \[\begin{split}
% %         A A^* =
% %         \begin{pmatrix}
% %             a_{11} & \dots & a_{1n} \\
% %             \vdots & \ddots & \vdots \\
% %             a_{n1} & \dots & a_{nn} \\
% %         \end{pmatrix} \cdot
% %         \begin{pmatrix}
% %             A_{11} & \dots & A_{n1} \\
% %             \vdots & \ddots & \vdots \\
% %             A_{1n} & \dots & A_{nn} \\
% %         \end{pmatrix} 
% % &=  \begin{pmatrix}
% %             \det A  & \dots & 0 \\
% %                   0 & \ddots & 0 \\
% %                   0 & \dots & \det A \\
% %         \end{pmatrix}
% %         \\
% %         &=\det A \cdot
% %         \begin{pmatrix}
% %             1 & \dots & 0\\
% %             0 & \ddots & 0\\
% %             0 & \dots & 1
% %         \end{pmatrix} = \det A \cdot I.
% %     \end{split}\]

% %     Taigi, $A \cdot A^* = \det A \cdot I$. Padaliję abi tapatybės puses iš skaičiaus $\det A \ne 0$, gauname $A \cdot \left(\frac{1}{\det A}A^*\right) = I$. Panašiai galime parodyti, kad $\left(\frac{1}{\det A}A^*\right) \cdot A = I$. Vadinasi, $\frac{1}{\det A}A^*$ yra matricos $A$ atvirkštinė, taigi $A$ yra neišsigimusi.
% %     \end{proof}

%     \bigskip

%     Atkreipsime dėmesį, kad matematinės aplinkos numeruojamos automatiškai. Kaip ir formules, matematinius apibrėžimus, teiginius, pavyzdžius galima cituoti kitur tekste pirma įvardijant su \texttt{\textbackslash label} o tada reikiamoje vietoje sukuriant citavimo nuorodą \texttt{\textbackslash ref}. Pavyzdžiui, mes galime pacituoti Teoremą \ref{teor1} arba Lemą \ref{lem1} tose teksto vietose, kuriose mums reikia jomis pasiremti.

% \subsection{Lentelės}

% Jei yra pristatomos lentelės, tai lentelių nuorodos turėtų būti paminėti tekste, pavyzdžiui: \ref{tab:xydata} lentelėje matomi kažkokie rezultatai.

% \begin{table}[H]
%     \centering
%     \caption{Lentelės numeruojamos viršuje, antraštės rašomos viršuje}
%     \begin{tabular}{|c|c|c|}
%         \hline
%         Stulpelis 1 & Stulpelis 2 & Stulpelis 3 \\
%         \hline
%          & &  \\
%          \hline
%          & & \\
%          \hline
%     \end{tabular}
%     \label{tab:xydata}
% \end{table}


% Kiekviena lentelė būtinai turi turėti pavadinimą, kuris, kaip ir lentelės numeris, rašomas toje pačioje eilutėje virš lentelės. Visos lentelės numeruojamos paeiliui (nerekomenduojama numeruoti raidėmis, pvz., 7 a lentelė).

% \subsection{Paveikslėliai, grafikai, diagramos, nuotraukos}
% Jei darbe naudojami paveikslėliai, būtina juos paminėti tekste, pvz.:~\ref{fig:grafikas1} paveikslėlyje~matome paveikslėlio pateikimo pavyzdį.

% \begin{figure}[H]
%     \centering
%     \includegraphics[width=0.5\textwidth]{images/AIC.png}
%     \caption{Paveikslėlių numeriai rašomi apačioje, antraštė rašoma apačioje}
%     \label{fig:grafikas1}
% \end{figure}


% Toliau eina tekstas po paveikslėliu.

% \subsection{Sąrašai}

% Nenumeruojamo sąrašo pavyzdys:
% \begin{itemize}
%     \item pirmasis elementas;
%     \item antrasis elementas.
% \end{itemize}

% Numeruojamo sąrašo pavydzys:
% \begin{enumerate}
%     \item lorem ipsum dolor sit amet;
%     \item consectetur adipiscing elit;
%     \item vivamus a nisl gravida.
% \end{enumerate}


% \section{Programinio kodo pateikimas}
% Šiame skyriuje pateikiamas programinio kodo pateikimo būdas rašto darbe.

% \subsection{Algoritmai}

% Algoritmai, lygiai taip pat, kaip ir paveikslėliai ar lentelės, yra numeruojami.
% Juos būtina paminėti tekste, pvz.:~\ref{alg:gd} naudojamas surasti minimalią funkcijos $\mathcal{L}$ reikšmę.

% \begin{algorithm}[h]
% \caption{Gradientinio nusileidimo pseudokodas}\label{alg:gd}
% \begin{algorithmic}[1]
%     \STATE \textcolor{blue}{\texttt{\textbf{\# Darome prielaidą, kad $\mathcal{L}$ apibrėžtas tekste}}}
%         \STATE Įeitis: $\mathcal{D}$ -- duomenų rinkinys
%         \STATE Įeitis: $\theta_0$ -- parametrų atsitiktinių reikšmių inicializavimas
%         \STATE Įeitis: $\gamma$ -- žingsnio dydis, mokymosi greitis (angl.~\textit{learning rate}, \textit{step size})
%         \STATE Įeitis: $m$ -- epochų skaičius
%         \FOR{$i = 1, 2, \dots, m$}
%             \STATE $\theta_i \coloneq \theta_{i-1} - \gamma \nabla_\theta \mathcal{L}(\mathcal{D}, \theta_{i-1})$
%             \STATE \textcolor{blue}{\texttt{\textbf{\# Funkcijos $\mathcal{L}$ išvestinė suskaičiuojama automatiškai, autograd pagalba}}}
%         \ENDFOR
% \end{algorithmic}
% \end{algorithm}

% \subsubsection{Skyrelio pavyzdys}
% \noindent Nebūtina naudoti daug skyrelių (\textit{subsubsections}).

\sectionnonum{Rezultatai}
\textcolor{red}{
    Detaliau, kas turi būti parašyta šiame skyriuje, rasite atitinkamos programos metodiniuose reikalavimuose. 
}


\sectionnonum{Išvados}
\textcolor{red}{
    Detaliau, kas turi būti parašyta šiame skyriuje, rasite atitinkamos programos metodiniuose reikalavimuose.  
}

% \printbibliography[title = {Šaltiniai}]


% \appendix
% \renewcommand{\thesection}{\arabic{section} priedas.}

% \section{\phantom{Priedas} Citavimo pavyzdžiai}
% Dokumente - \textit{bibliografija.bib}, reikia sudėti visus cituojamus šaltinius ir panaudojus funkciją \textit{\{\textbackslash cite\{cituojamo objekto pavadinimas\}\}} atitinkamas šaltinis bus pridėtas prie literatūros šaltinių sąrašo.


% \textit{bibliografija.bib} galima rasti kelių dažniausiai cituojamų šaltinių tipų pavyzdžius:
% \begin{itemize}
%     \item internetiniai puslapiai (\textit{@online}) \cite{PvzInternetinisPuslapis},
%     \item duomenų rinkiniai (\textit{@dataset}) \cite{dataset}
%     \item straipsniai (\textit{@article}) \cite{PvzStraipsnLt, PvzStraipsnEn}, 
%     \item straipsniai iš konferencijos (\textit{@inproceedings}) \cite{PvzKonfLt, PvzKonfEn}, 
%     \item knygos (\textit{@book}) \cite{PvzKnygLt, PvzKnygEn}, 
%     \item baigiamieji darbai (\textit{@thesis arba mastersthesis/phdthesis} \cite{PvzMagistrLt, PvzPhdEn})
%     \item elektroninės publikacijos (\textit{@misc}) \cite{PvzElPubLt, PvzElPubEn}
% \end{itemize}

% Taip pat yra pateikti pavyzdžiai - ChatGPT citavimui, tiek bendrai\cite{chatgpt_bendrai}, tiek konkrečiam pokalbiui\cite{chatgpt_pokalbis}.

\end{document}
