%%%%%
%%%%%  Naudokite LUALATEX, ne LATEX.
%%%%%
%%%%
\documentclass[]{VUMIFTemplateClass}

\usepackage{indentfirst}
\usepackage{amsmath, amsthm, amssymb, amsfonts}
\usepackage{mathtools}
\usepackage{physics}
\usepackage{graphicx}
\usepackage{verbatim}
\usepackage[hidelinks]{hyperref}
\usepackage{color,algorithm,algorithmic}
\usepackage[nottoc]{tocbibind}
\usepackage{tocloft}
\usepackage{titlesec}

\makeatletter
\renewcommand{\fnum@algorithm}{\thealgorithm}
\makeatother
\renewcommand\thealgorithm{\arabic{algorithm} algoritmas}

\usepackage{biblatex}
\bibliography{bibliografija}
%% norint pakeisti bibliografijos šaltinių numeravimą (skaitiniu arba raidiniu), pakeitimus atlikti VUMIFTemplateClass.cls 150 eilutėje

% Author's MACROS
\newcommand{\EE}{\mathbb{E}\,} % Mean
\newcommand{\ee}{{\mathrm e}}  % nice exponent
\newcommand{\RR}{\mathbb{R}}


\studijuprograma{Programų sistemų}
\darbotipas{Bakalauro baigiamojo darbo
planas}
\darbopavadinimas{Likvidavimo algoritmo tobulinimas perviršinio užstato skolinimo protokoluose}
\darbopavadinimasantras{Improving liquidation algorithms in over-collateralized lending protocols}
\autorius{Vismantas Stonkus}

\vadovas{prof. dr. Remigijus Paulavičius}
\recenzentas{prof. dr. Saulius Masteika}

\begin{document}
\selectlanguage{lithuanian}

\onehalfspacing
\begin{titlepage}
\vskip 20pt
\begin{center}
\includegraphics[scale=0.55]{images/MIF.png}
\end{center}

\makeatletter

\vskip 20pt
\centerline{\bf \large \textbf{VILNIAUS UNIVERSITETAS}}
\vskip 10pt
\centerline{\large \textbf{MATEMATIKOS IR INFORMATIKOS FAKULTETAS}}
\vskip 10pt
\centerline{\large \textbf{\MakeUppercase{\@studijuprograma \space studijų programa}}}

\vskip 80pt
\centerline{\Large \@darbotipas}
\vskip 20pt
\vskip 80pt

\centering{
    \begin{tabular}{rcp{.7\textwidth}}
        {\Large Praktiką atliko} & {\Large :} & {\Large \@autorius}\\[10pt]
        {\Large Universiteto praktikos vadovas} & {\Large :} & {\Large \@vadovas}\\[10pt]
        {\Large Praktikos institucija} & {\Large :} & {\Large \@institucija}\\[10pt]
        {\Large Organizacijos praktikos vadovas} & {\Large :} & {\Large \@institucijosVadovas}\\[10pt]
        {} & {} & {\Large Technologijų vadovas (CTO)}\\[10pt]
        {\Large Organizacijos praktikos vadovo įvertinimas} & {\Large :} & {\Large \@institucijosIvertinimas}\\[10pt]
    \end{tabular}}

\vskip 110pt

\centerline{\large \textbf{Vilnius}}
\centerline{\large \textbf{\the\year{}}}

\makeatother

\newpage
\end{titlepage}
%\newgeometry{top=2cm,bottom=2cm,right=2cm,left=3cm}
\setcounter{page}{2}


% darbo planas, kuriame pateikiami tyrimo objektas ir aktualumas, darbo tikslas, keliami uždaviniai ir laukiami rezultatai, tyrimo metodai, numatomas darbo atlikimo procesas apibūdinami darbui aktualūs literatūros šaltiniai.
% Pastaba. Darbo uždavinyje apibrėžiamas siekiamas rezultatas, kad būtų galimybė išmatuoti,
% ar tikslai ir uždaviniai yra išspręsti, bei kokiu lygiu (vertinant kiekybę bei kokybę). Pavyzdžiui, „Atlikti literatūros .... analizę“ nėra tinkamas uždavinys, nes nusako procesą, tačiau
% neapibrėžia jo rezultato. Tinkamos uždavinio formuluotės šablonai: „Išanalizuoti literatūrą
% … siekiant apžvelgti ir įvertinti /… metodų tinkamumą sprendžiamam uždaviniui/privalumus ir trūkumus sprendžiant … uždavinį/rekomenduojamas ... projektavimo gaires, šablonus
% ir pan.“

% Bakalauriniame darbe bus vystomas toliau kursinis darbas todėl bus pernaudota didžioji dali kodo susijusia su duomenų rinkimu.

\section{Tyrimo objektas ir aktualumas}

Šiame darbe bus tiriami kriptovaliutų paskolų platformų likvidavimo mechanizmai, ypatingą dėmesį skiriant \textit{Venus} protokolui, veikiančiam \textit{Binance Smart Chain} (BSC) blokų grandinėje. Bus analizuojamos likvidavimo proceso ypatybės ir ieškoma būdų jį optimizuoti. Šio protokolo veikimo principai būdingi daugeliui kitų perviršinio užstato skolinimo sistemų. Efektyvesnis likvidavimo algoritmas, padidinantis likvidatoriaus pelną, galėtų būti pritaikytas ir kitose decentralizuotose skolinimo platformose, siekiant optimizuoti likvidacijos procesą.

\section{Darbo tikslas}

Darbo tikslas – sukurti ir optimizuoti likvidavimo algoritmą, kuris maksimaliai padidintų likvidatoriaus pelną.

\section{Keliami uždaviniai ir laukiami rezultatai}

Užsibrėžtam tikslui pasiekti keliami šie uždaviniai:
\begin{enumerate}
  \item Apžvelgti esamus perviršinio užstato skolinimo protokolus, išanalizuoti jų veikimo principus ir palyginti juos su \textit{Venus} protokolu. Kadangi daugelis skolinimo protokolų kriptovaliutų ekosistemoje veikia panašiai, jei ši prielaida pasitvirtins, gauti rezultatai galės būti pritaikyti ir kituose protokoluose.
  \item Išsamiai išnagrinėti \textit{Venus} protokolo likvidavimo mechanizmą.
  \item Sukurti ir/arba modifikuoti efektyvesnį likvidavimo algoritmą. Šiuo uždaviniu siekiame praplėsti kursiniame darbe pristatytas likvidavimo strategijas („atkartoti“, „iki uždarymo ribos“, „pilnas išeikvojimas“) papildant jas keturiomis naujomis strategijomis, kurios atsisako fiksuotų skolos ir užstato valiutų porų:
  \begin{itemize}
    \item \textbf{Didžiausia skola} (single largest borrow) – grąžinama ta skolos valiuta, kurios suma yra didžiausia. Užstatas pasirenkamas iš tos pačios valiutos, jei užstato kiekis yra pakankamas, kitu atveju – valiuta, kurios vertė skolininko portfelyje didžiausia.
    \item \textbf{Nuo mažiausio likvidavimo slenksčio} (from smallest collateral factor) – pirmiausia pasirenkamos užstato valiutos su mažiausiu likvidavimo slenksčiu, o paskutinė likvidacija vykdoma pagal „Didžiausios skolos“ strategiją.
    \item \textbf{Nuo didžiausio likvidavimo slenksčio} (from largest collateral factor) – analogiška ankstesnei strategijai, tačiau prioritetas teikiamas užstato valiutoms su didžiausiu likvidavimo slenksčiu.
    \item \textbf{Vienodos valiutos} (same tokens) – atliekamos „Pilno išeikvojimo“ strategijos likvidacijos tik toms valiutoms, kurios yra tiek užstatytos, tiek pasiskolintos. Ši strategija leidžia sumažinti valiutų keitimo riziką ir likvidumo problemas.
  \end{itemize}
  \item Palyginti sukurtas strategijas tarpusavyje bei su istorinėmis likvidacijomis.
  \item Apibendrinti rezultatus ir pateikti išvadas bei rekomendacijas. Bus bandoma atsakyti į šiuos klausimus:
  \begin{itemize}
    \item Ar „Nuo mažiausio likvidavimo slenksčio“ strategija yra pelningiausia likvidatoriui? Ši hipotezė grindžiama skolinimosi pajėgumo formule: jis priklauso nuo užstato valiutų verčių ir jų likvidavimo slenksčių sandaugų sumos. Tarkime, kad skolininkas yra užstatęs dvi vienodos vertės valiutas, tačiau vienos valiutos likvidavimo slenkstis yra 50\%, o kitos – 90\%. Užstatas su mažesniu likvidavimo slenksčiu mažiau prisideda prie skolinimosi pajėgumo nei užstatas su didesniu slenksčiu. Todėl likviduojant pirmiausia mažesnio slenksčio užstatą, mažiau paveikiamas bendras skolinimosi pajėgumas, o tai gali lemti didesnį likvidatoriaus pelną.
    \item Ar „Nuo didžiausio likvidavimo slenksčio“ strategija, kuri veikia priešingai nei optimalios strategijos, rodys mažesnį pelną likvidatoriui?
    \item Ar „Vienodos valiutos“ strategija sustiprina „Pilno išeikvojimo“ strategijos pagrįstumą, nes eliminuoja valiutų konvertavimo riziką ir likvidumo trūkumą vykdant arbitražą didelėmis sumomis?
  \end{itemize}
\end{enumerate}

\section{Tyrimo metodai}

Tyrimas bus atliekamas analizuojant istorinius \textit{Venus} protokolo likvidavimo duomenis, siekiant įvertinti, kaip skirtingos strategijos galėtų būti pritaikytos realiose situacijose.

\begin{enumerate}
  \item Istorinių duomenų analizė:
  \begin{itemize}
    \item Kiekviena tirta istorinė likvidacija bus simuliuojama naujomis strategijomis, palyginant skirtingų algoritmų pelningumą.
    \item Bus išlaikoma skolininko tapatybė, t. y. analizė bus atliekama su tais pačiais skolininkais, kurie buvo likviduoti istorinėse transakcijose.
    \item Likvidacijos rezultatų būsena nebus perkelta tarp skirtingų įvykių – kiekviena strategija bus testuojama individualiai, be poveikio likusioms likvidacijoms ateityje.
  \end{itemize}

  \item Simuliacijų vykdymas:
  \begin{itemize}
    \item Simuliacijos bus atliekamos Forge testavimo aplinkoje (Ethereum pagrindu sukurtas blokų grandinės testavimo karkasas).
    \item Likvidavimo dydžiai ir strategijos bus modeliuojami naudojant Solidity programavimo kalbą.
  \end{itemize}

  \item Pelno skaičiavimas:
  \begin{itemize}
    \item Likvidacijos pelnas apskaičiuojamas įvertinant gautą užstatą, atėmus grąžintą paskolos sumą.
    \item Atsižvelgiama į blokų grandinės transakcijų mokesčius.
    \item Valiutų kainos bus imamos iš \textit{Venus} protokolo naudojamos orakulo sistemos, kuri atsakinga už valiutų vertės nustatymą likvidavimo metu.
  \end{itemize}
\end{enumerate}

\section{Numatomas darbo atlikimo procesas}

\begin{enumerate}
  \item Istorinių duomenų surinkimas.
   \begin{itemize}
    \item Gauti visų \textit{Venus} platformos likvidacijų duomenis iš BSC blokų grandinės.
  \end{itemize}

  \item Naujų strategijų realizavimas.
   \begin{itemize}
    \item Implementuoti keturias papildomas strategijas, kurios atsisako fiksuotų skolos ir užstato valiutų porų: „Didžiausia skola“, „Nuo mažiausio likvidavimo slenksčio“, „Nuo didžiausio likvidavimo slenksčio“, „Vienodos valiutos“.
  \end{itemize}

  \item Strategijų palyginimas ir rezultatų analizė.
  \begin{itemize}
   \item Vykdyti istorinių likvidacijų simuliacijas su skirtingomis strategijomis.
   \item Lyginti pelningumą, efektyvumą bei strategijų poveikį skolininko sveikumo koeficientui.
  \end{itemize}

  \item Išvados ir ateities darbai.
  \begin{itemize}
   \item Apibendrinti tyrimo rezultatus, pateikti rekomendacijas dėl optimalios likvidavimo strategijos pasirinkimo.
   \item Įvertinti galimus būdus, kaip optimizuoti likvidavimo procesą \textit{Venus} ir kituose skolinimo protokoluose.
  \end{itemize}

\end{enumerate}

\nocite{*}

\printbibliography[title = {Literatūra ir šaltiniai}]

\end{document}
